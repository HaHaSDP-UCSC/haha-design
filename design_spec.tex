\documentclass[journal,compsoc]{IEEEtran}
%
% If IEEEtran.cls has not been installed into the LaTeX system files,
% manually specify the path to it like:
% \documentclass[journal]{../sty/IEEEtran}

\usepackage{listings} % For code box
\usepackage{array} % For first column bold in table
\usepackage{parskip} % Disable indentation, adds spacing
\usepackage{rotating} % for sideways table
\usepackage{float}
\usepackage{comment} % for block commenting
\usepackage{mathtools} % for using mathematical symbols
\usepackage[justification=centering]{caption} % for centering captions
\usepackage{stfloats}
\usepackage{lcd}
\usepackage{subcaption}

% Some very useful LaTeX packages include:
% (uncomment the ones you want to load)


% *** MISC UTILITY PACKAGES ***
%
%\usepackage{ifpdf}
% Heiko Oberdiek's ifpdf.sty is very useful if you need conditional
% compilation based on whether the output is pdf or dvi.
% usage:
% \ifpdf
%   % pdf code
% \else
%   % dvi code
% \fi
% The latest version of ifpdf.sty can be obtained from:
% http://www.ctan.org/pkg/ifpdf
% Also, note that IEEEtran.cls V1.7 and later provides a builtin
% \ifCLASSINFOpdf conditional that works the same way.
% When switching from latex to pdflatex and vice-versa, the compiler may
% have to be run twice to clear warning/error messages.






% *** CITATION PACKAGES ***
%
%\usepackage{cite}
% cite.sty was written by Donald Arseneau
% V1.6 and later of IEEEtran pre-defines the format of the cite.sty package
% \cite{} output to follow that of the IEEE. Loading the cite package will
% result in citation numbers being automatically sorted and properly
% "compressed/ranged". e.g., [1], [9], [2], [7], [5], [6] without using
% cite.sty will become [1], [2], [5]--[7], [9] using cite.sty. cite.sty's
% \cite will automatically add leading space, if needed. Use cite.sty's
% noadjust option (cite.sty V3.8 and later) if you want to turn this off
% such as if a citation ever needs to be enclosed in parenthesis.
% cite.sty is already installed on most LaTeX systems. Be sure and use
% version 5.0 (2009-03-20) and later if using hyperref.sty.
% The latest version can be obtained at:
% http://www.ctan.org/pkg/cite
% The documentation is contained in the cite.sty file itself.






% *** GRAPHICS RELATED PACKAGES ***
%
\ifCLASSINFOpdf
  % \usepackage[pdftex]{graphicx}
  % declare the path(s) where your graphic files are
  % \graphicspath{{../pdf/}{../jpeg/}}
  % and their extensions so you won't have to specify these with
  % every instance of \includegraphics
  % \DeclareGraphicsExtensions{.pdf,.jpeg,.png}
\else
  % or other class option (dvipsone, dvipdf, if not using dvips). graphicx
  % will default to the driver specified in the system graphics.cfg if no
  % driver is specified.
  % \usepackage[dvips]{graphicx}
  % declare the path(s) where your graphic files are
  % \graphicspath{{../eps/}}
  % and their extensions so you won't have to specify these with
  % every instance of \includegraphics
  % \DeclareGraphicsExtensions{.eps}
\fi
% graphicx was written by David Carlisle and Sebastian Rahtz. It is
% required if you want graphics, photos, etc. graphicx.sty is already
% installed on most LaTeX systems. The latest version and documentation
% can be obtained at:
% http://www.ctan.org/pkg/graphicx
% Another good source of documentation is "Using Imported Graphics in
% LaTeX2e" by Keith Reckdahl which can be found at:
% http://www.ctan.org/pkg/epslatex
%
% latex, and pdflatex in dvi mode, support graphics in encapsulated
% postscript (.eps) format. pdflatex in pdf mode supports graphics
% in .pdf, .jpeg, .png and .mps (metapost) formats. Users should ensure
% that all non-photo figures use a vector format (.eps, .pdf, .mps) and
% not a bitmapped formats (.jpeg, .png). The IEEE frowns on bitmapped formats
% which can result in "jaggedy"/blurry rendering of lines and letters as
% well as large increases in file sizes.
%
% You can find documentation about the pdfTeX application at:
% http://www.tug.org/applications/pdftex





% *** MATH PACKAGES ***
%
%\usepackage{amsmath}
% A popular package from the American Mathematical Society that provides
% many useful and powerful commands for dealing with mathematics.
%
% Note that the amsmath package sets \interdisplaylinepenalty to 10000
% thus preventing page breaks from occurring within multiline equations. Use:
%\interdisplaylinepenalty=2500
% after loading amsmath to restore such page breaks as IEEEtran.cls normally
% does. amsmath.sty is already installed on most LaTeX systems. The latest
% version and documentation can be obtained at:
% http://www.ctan.org/pkg/amsmath



% *** SPECIALIZED LIST PACKAGES ***
%
%\usepackage{algorithmic}
% algorithmic.sty was written by Peter Williams and Rogerio Brito.
% This package provides an algorithmic environment fo describing algorithms.
% You can use the algorithmic environment in-text or within a figure
% environment to provide for a floating algorithm. Do NOT use the algorithm
% floating environment provided by algorithm.sty (by the same authors) or
% algorithm2e.sty (by Christophe Fiorio) as the IEEE does not use dedicated
% algorithm float types and packages that provide these will not provide
% correct IEEE style captions. The latest version and documentation of
% algorithmic.sty can be obtained at:
% http://www.ctan.org/pkg/algorithms
% Also of interest may be the (relatively newer and more customizable)
% algorithmicx.sty package by Szasz Janos:
% http://www.ctan.org/pkg/algorithmicx




% *** ALIGNMENT PACKAGES ***
%
%\usepackage{array}
% Frank Mittelbach's and David Carlisle's array.sty patches and improves
% the standard LaTeX2e array and tabular environments to provide better
% appearance and additional user controls. As the default LaTeX2e table
% generation code is lacking to the point of almost being broken with
% respect to the quality of the end results, all users are strongly
% advised to use an enhanced (at the very least that provided by array.sty)
% set of table tools. array.sty is already installed on most systems. The
% latest version and documentation can be obtained at:
% http://www.ctan.org/pkg/array


% IEEEtran contains the IEEEeqnarray family of commands that can be used to
% generate multiline equations as well as matrices, tables, etc., of high
% quality.




% *** SUBFIGURE PACKAGES ***
%\ifCLASSOPTIONcompsoc
%  \usepackage[caption=false,font=normalsize,labelfont=sf,textfont=sf]{subfig}
%\else
%  \usepackage[caption=false,font=footnotesize]{subfig}
%\fi
% subfig.sty, written by Steven Douglas Cochran, is the modern replacement
% for subfigure.sty, the latter of which is no longer maintained and is
% incompatible with some LaTeX packages including fixltx2e. However,
% subfig.sty requires and automatically loads Axel Sommerfeldt's caption.sty
% which will override IEEEtran.cls' handling of captions and this will result
% in non-IEEE style figure/table captions. To prevent this problem, be sure
% and invoke subfig.sty's "caption=false" package option (available since
% subfig.sty version 1.3, 2005/06/28) as this is will preserve IEEEtran.cls
% handling of captions.
% Note that the Computer Society format requires a larger sans serif font
% than the serif footnote size font used in traditional IEEE formatting
% and thus the need to invoke different subfig.sty package options depending
% on whether compsoc mode has been enabled.
%
% The latest version and documentation of subfig.sty can be obtained at:
% http://www.ctan.org/pkg/subfig




% *** FLOAT PACKAGES ***
%
%\usepackage{fixltx2e}
% fixltx2e, the successor to the earlier fix2col.sty, was written by
% Frank Mittelbach and David Carlisle. This package corrects a few problems
% in the LaTeX2e kernel, the most notable of which is that in current
% LaTeX2e releases, the ordering of single and double column floats is not
% guaranteed to be preserved. Thus, an unpatched LaTeX2e can allow a
% single column figure to be placed prior to an earlier double column
% figure.
% Be aware that LaTeX2e kernels dated 2015 and later have fixltx2e.sty's
% corrections already built into the system in which case a warning will
% be issued if an attempt is made to load fixltx2e.sty as it is no longer
% needed.
% The latest version and documentation can be found at:
% http://www.ctan.org/pkg/fixltx2e


%\usepackage{stfloats}
% stfloats.sty was written by Sigitas Tolusis. This package gives LaTeX2e
% the ability to do double column floats at the bottom of the page as well
% as the top. (e.g., "\begin{figure*}[!b]" is not normally possible in
% LaTeX2e). It also provides a command:
%\fnbelowfloat
% to enable the placement of footnotes below bottom floats (the standard
% LaTeX2e kernel puts them above bottom floats). This is an invasive package
% which rewrites many portions of the LaTeX2e float routines. It may not work
% with other packages that modify the LaTeX2e float routines. The latest
% version and documentation can be obtained at:
% http://www.ctan.org/pkg/stfloats
% Do not use the stfloats baselinefloat ability as the IEEE does not allow
% \baselineskip to stretch. Authors submitting work to the IEEE should note
% that the IEEE rarely uses double column equations and that authors should try
% to avoid such use. Do not be tempted to use the cuted.sty or midfloat.sty
% packages (also by Sigitas Tolusis) as the IEEE does not format its papers in
% such ways.
% Do not attempt to use stfloats with fixltx2e as they are incompatible.
% Instead, use Morten Hogholm'a dblfloatfix which combines the features
% of both fixltx2e and stfloats:
%
% \usepackage{dblfloatfix}
% The latest version can be found at:
% http://www.ctan.org/pkg/dblfloatfix




%\ifCLASSOPTIONcaptionsoff
%  \usepackage[nomarkers]{endfloat}
% \let\MYoriglatexcaption\caption
% \renewcommand{\caption}[2][\relax]{\MYoriglatexcaption[#2]{#2}}
%\fi
% endfloat.sty was written by James Darrell McCauley, Jeff Goldberg and
% Axel Sommerfeldt. This package may be useful when used in conjunction with
% IEEEtran.cls'  captionsoff option. Some IEEE journals/societies require that
% submissions have lists of figures/tables at the end of the paper and that
% figures/tables without any captions are placed on a page by themselves at
% the end of the document. If needed, the draftcls IEEEtran class option or
% \CLASSINPUTbaselinestretch interface can be used to increase the line
% spacing as well. Be sure and use the nomarkers option of endfloat to
% prevent endfloat from "marking" where the figures would have been placed
% in the text. The two hack lines of code above are a slight modification of
% that suggested by in the endfloat docs (section 8.4.1) to ensure that
% the full captions always appear in the list of figures/tables - even if
% the user used the short optional argument of \caption[]{}.
% IEEE papers do not typically make use of \caption[]'s optional argument,
% so this should not be an issue. A similar trick can be used to disable
% captions of packages such as subfig.sty that lack options to turn off
% the subcaptions:
% For subfig.sty:
% \let\MYorigsubfloat\subfloat
% \renewcommand{\subfloat}[2][\relax]{\MYorigsubfloat[]{#2}}
% However, the above trick will not work if both optional arguments of
% the \subfloat command are used. Furthermore, there needs to be a
% description of each subfigure *somewhere* and endfloat does not add
% subfigure captions to its list of figures. Thus, the best approach is to
% avoid the use of subfigure captions (many IEEE journals avoid them anyway)
% and instead reference/explain all the subfigures within the main caption.
% The latest version of endfloat.sty and its documentation can obtained at:
% http://www.ctan.org/pkg/endfloat
%
% The IEEEtran \ifCLASSOPTIONcaptionsoff conditional can also be used
% later in the document, say, to conditionally put the References on a
% page by themselves.




% *** PDF, URL AND HYPERLINK PACKAGES ***
%
%\usepackage{url}
% url.sty was written by Donald Arseneau. It provides better support for
% handling and breaking URLs. url.sty is already installed on most LaTeX
% systems. The latest version and documentation can be obtained at:
% http://www.ctan.org/pkg/url
% Basically, \url{my_url_here}.




% *** Do not adjust lengths that control margins, column widths, etc. ***
% *** Do not use packages that alter fonts (such as pslatex).         ***
% There should be no need to do such things with IEEEtran.cls V1.6 and later.
% (Unless specifically asked to do so by the journal or conference you plan
% to submit to, of course. )


% correct bad hyphenation here
\hyphenation{op-tical net-works semi-conduc-tor}


\begin{document}
%
% paper title
% Titles are generally capitalized except for words such as a, an, and, as,
% at, but, by, for, in, nor, of, on, or, the, to and up, which are usually
% not capitalized unless they are the first or last word of the title.
% Linebreaks \\ can be used within to get better formatting as desired.
% Do not put math or special symbols in the title.
\title{
  Help Alert --- Home Assistance (HA-HA) Button \\
  Design Specification
}
%
%
% author names and IEEE memberships
% note positions of commas and nonbreaking spaces ( ~ ) LaTeX will not break
% a structure at a ~ so this keeps an author's name from being broken across
% two lines.
% use \thanks{} to gain access to the first footnote area
% a separate \thanks must be used for each paragraph as LaTeX2e's \thanks
% was not built to handle multiple paragraphs
%

\author{Jamielynne~Batugo, Marco~Carmona, Conrad~Christensen, Jake~Lee, Kevin~Lee, Brian~Nichols, Jesus~Soto, August~Valera, and Jeffrey~Zheng}% <-this % stops a space

\begin{comment}
\thanks{Jamielynne~Batugo, Marco~Carmona, Conrad~Christensen, Jake~Lee, Kevin~Lee, Brian~Nichols, Jesus~Soto, August~Valera, and Jeffrey~Zheng were students at the University of California, Santa Cruz}% <-this % stops a space
\thanks{Ali Adabi, Patrick Mantey, Stephen Petersen, and Anujan Varma were professors and mentors at the University of California, Santa Cruz}% <-this % stops a space
\end{comment}
\thanks{The full source of this project is released under the BSD 2-Clause Licence.}%
\thanks{Manuscript received \today}

% note the % following the last \IEEEmembership and also \thanks -
% these prevent an unwanted space from occurring between the last author name
% and the end of the author line. i.e., if you had this:
%
% \author{....lastname \thanks{...} \thanks{...} }
%                     ^------------^------------^----Do not want these spaces!
%
% a space would be appended to the last name and could cause every name on that
% line to be shifted left slightly. This is one of those "LaTeX things". For
% instance, "\textbf{A} \textbf{B}" will typeset as "A B" not "AB". To get
% "AB" then you have to do: "\textbf{A}\textbf{B}"
% \thanks is no different in this regard, so shield the last } of each \thanks
% that ends a line with a % and do not let a space in before the next \thanks.
% Spaces after \IEEEmembership other than the last one are OK (and needed) as
% you are supposed to have spaces between the names. For what it is worth,
% this is a minor point as most people would not even notice if the said evil
% space somehow managed to creep in.



% The paper headers
\markboth{University of California, Santa Cruz --- Senior Design Project, 2017}%
{Shell \MakeLowercase{\textit{et al.}}: Bare Demo of IEEEtran.cls for IEEE Journals}
% The only time the second header will appear is for the odd numbered pages
% after the title page when using the twoside option.
%
% *** Note that you probably will NOT want to include the author's ***
% *** name in the headers of peer review papers.                   ***
% You can use \ifCLASSOPTIONpeerreview for conditional compilation here if
% you desire.




% If you want to put a publisher's ID mark on the page you can do it like
% this:
%\IEEEpubid{0000--0000/00\$00.00~\copyright~2015 IEEE}
% Remember, if you use this you must call \IEEEpubidadjcol in the second
% column for its text to clear the IEEEpubid mark.



% use for special paper notices
%\IEEEspecialpapernotice{(Invited Paper)}




% make the title area
\maketitle

% As a general rule, do not put math, special symbols or citations
% in the abstract or keywords.
\begin{abstract}
  Current senior emergency button systems on the market serve a very specific user base, immediately escalating any request for assistance to emergency services (police, fire, EMS), and require a reoccurring cost to maintain connectivity to staffed call centers to handle these requests. There is a need for a cheap, single upfront cost device that simply calls your friends/neighbors to help you in times of non-critical emergencies. This project aims to build such a system, composed of a series of button wearables and base stations, to be deployed and tested within the De Anza Santa Cruz retirement community.
\end{abstract}

% Note that keywords are not normally used for peerreview papers.
\begin{IEEEkeywords}
  assistance, bluetooth, button, elderly, emergency, seniors, Xbee,
\end{IEEEkeywords}






% For peer review papers, you can put extra information on the cover
% page as needed:
% \ifCLASSOPTIONpeerreview
% \begin{center} \bfseries EDICS Category: 3-BBND \end{center}
% \fi
%
% For peerreview papers, this IEEEtran command inserts a page break and
% creates the second title. It will be ignored for other modes.
\IEEEpeerreviewmaketitle

\section{Introduction}
\IEEEPARstart{T}{he} Help Alert --- Home Assistance system is composed of a base station and a wirelessly connected button.  The base station is wirelessly connected to other base stations to facilitate long-range communications over a mesh network, in order to forgo the requirement of having a pre-existing communications infrastructure.

\subsection{Motivation}
One of the primary principles driving this project is cost.  It needs be affordable and without subscription fees.  As a result, the hardware has been chosen primarily to comply with this cost constraint, which in turn required a sizable portion of Winter quarter to be dedicated in researching different hardware platforms.

\subsection{Objectives}
The fundamental idea behind the Home Assistance Help Alert System is to provide an affordable means of protecting the health, welfare, and safety of elderly people through an active ad-hoc wireless network estimated to have a working range of 300 feet, specifically for residents of the De Anza Santa Cruz Senior Citizen Community.  The final design should consider environmental conditions of the location (interference, transmission range) and consist of parts which can be obtained or reproduced by a third party at scale.  The final design must also be affordable and have a form factor satisfactory to the end user.  A detailed design document shall be made publicly available for non-profit entities who would like to expand upon the project.

\section{Hardware Selection}

\subsection{Base Station}

\subsubsection{Microcontroller}

\begin{table*}[t]
  \centering
  \begin{tabular}{>{\bfseries}l|l l l l}
    Device & TI CC430 & Kinetis KW30Z & TI CC2510/11DK & Atmel SAMB11 \\
    \hline
    Datasheet & Link & Link & Link & Link \\
    DevBoard Price & \$149 for 2 & \$145 for 2 & \$220 for 2 & \\
    Chip Price (x50) & \$6.64 & \$4.72 & \$4.16 & \$3.51 \\
    915 MHz Antenna & 2 (no Zigbee) & No & No & No, but integrates with \\
    2.4 GHz Antenna & No & Yes (Bluetooth, Zigbee) & Yes CC2500 & Yes \\
    Flash & 32 kB & 128-512 kB & 8/16/32 kB & 256 kB \\
    RAM & 4 kB & 64 kB & 1/2/4 kB & 128 kB \\
    Supply & 2.0-3.6 V & 3.6 V & 2.0-3.6 V & 2.3-3.6 V
  \end{tabular}
  \caption{Base Station Microcontroller Costs and Specifications}
\end{table*}

TODO: Missing figure

\subsubsection{Base Station Interconnect}

Wireless interconnection is a critical component governing the performance of our system.  A communication link between the button and the base station allows for a help alert signal to be transmitted locally within a small home environment, while the link between multiple base stations allows the signal to propagate further throughout a larger neighborhood.

\begin{table*}[t]
  \centering
  \begin{tabular}{l|l}
    Band & Description \\
    \hline
    315 US/433 EU MHz & Garage door opener/FOB, may be subject to some FCC problems, very low bandwidth \\
    915 US/868 EU MHz & ZigBee, unlicensed, free (mostly) \\
    2.4 GHz & Wi-Fi, Bluetooth, Zigbee, unlicensed, free (mostly), limited range
  \end{tabular}
  \caption{Frequency Band Options}
\end{table*}

Selection in broadcast frequency considers the limitations of transmitted power as well as form factors of potential antennas.  FCC regulations described in Title 47 Part 15 dictate the limitations of low-power, non-licensed, signal transmissions in the United States air space.  While most frequency bands are very limited for public use, there are several bands which are set aside as free for anyone to use under specific constraints; in comparison, broadcasts in “free” frequencies are allowed much greater transmission power to their restricted counterparts.  Additionally, electromagnetic waves are governed by an inverse relation between its frequency and length; a wave of increasing frequency has a decreasing wavelength and vice versa.  As a rule of thumb, an antenna’s size grows as the wavelength is increased for our signal of interest. Higher frequency signals also tend to have channels with larger bandwidths, allowing for more data to be sent with each transmission.  Due to the simple nature of the application, the high data rate is not critical.

Communication between the button and base station is projected to be at least 50 feet, while distances between base stations should be able to reach at least 300 meters (approximately 1000 feet).  To meet these specifications while keeping within FCC regulations, the “free” frequency bands will be used. Two of the most popular frequency bands include the 915 MHz band and the 2.4 GHz band.\\

\textbf {Radio Frequencies:}

\begin{LaTeXdescription}
  \item [2.4 GHz] will be used for the button-to-base communication. This band is hugely populated by many mainstream applications like WiFi and Bluetooth, but the small distance between the button and base will hopefully not be heavily affected by other traffic. Additionally, antennas in the 2.4 GHz range can be made compact, allowing the button to made into a small form factor.
  \item[915 MHz] will be used for base-to-base communications.  Antennas in this band will generally be slightly larger than the corresponding 2.4 GHz antennas, but because base stations are designed to be stationary in a person’s home, antenna size is no longer a limiting factor.  Furthermore, air space in the 915 MHz band is not as congested to its 2.4 GHz counterpart so the reduction in potential interference allows for longer transmissions to be more attainable.
\end{LaTeXdescription}

\textbf {Base-to-Base Protocols:}

\begin{LaTeXdescription}
  \item [DigiMesh] DigiMesh is a proprietary protocol that implements a variant of the AODV routing algorithm. Includes network self-configuration, self-healing, and sleep synchronization.
  \item [802.15.4  (Custom Protocol)] Writing our own mesh protocol on top of the 802.15.4 standard, which on its own only provides point to point connections (possibly with reliability, depending on implementation). This choice would be difficult and take significant amounts of time to accomplish in the time that we have.
  \item [ZigBee] A ZigBee device will come with a full stack implementation of a mesh networking protocol which will allow us to concentrate on the implementation of the actual device. It will also allow us to have the time to extensively test the platform to see the limits of the protocol.
  \item [Bluetooth Low Energy] Bluetooth Low Energy (BLE) is a point-to-point wireless communication technology.  Its power efficiency allows the user to implement a variety of services, making feedback possible such as when the button’s battery is low or detecting if the button is located near the base station.  These services allow the button to be smart and flexible when communicating.
\end{LaTeXdescription}

\textbf {DigiMesh features the following:}

\begin{LaTeXdescription}
  \item[Self-healing] any node may enter or leave the network at any time without causing the network as a whole to fail.
  \item[Peer-to-peer architecture] no hierarchy and no parent-child relationships are needed.
  \item[Quiet Protocol] routing overhead will be reduced by using a reactive protocol similar to AODV.
  \item[Route Discovery] rather than maintaining a network map, routes will be discovered and created only when needed.
  \item[Selective acknowledgments] only the destination node will reply to route requests.
  \item[Reliable delivery] reliable delivery of data is accomplished by means of acknowledgements.
  \item[Sleep Modes] low power sleep modes with synchronized wake up are supported, with variable sleep and wake up times.
\end{LaTeXdescription}

The DigiMesh protocol is a proprietary protocol and uses a variant of the AODV routing algorithm. Routing tables are built only for the immediate destinations a node has a need for.  Like wirelessHART, all nodes can be routers and have the option to sleep and wake on a synchronized schedule[6].

Routers for the network are determined as needed which results in routing table entries only for those routes in use. Unlike the other protocols there is no central coordinator and synchronization is accomplished through an election process.  Routing to nodes not directly connected to the sending node is accomplished by sending to destination specified in the routing table, which if does not exist the route is discovered through AODV and stored.

If a packet on its way to a destination encounters a node that was present in the routing table but no longer is failing at the time of sending, an AODV discovery is initiated and if the correct route if found through an alternate path the routing tables of all affected nodes will be updated and the packet will be delivered[6].  All packets are acknowledged with messages sent to the receiver.

Because every node has the ability to be a sleeping node power can be saved in between routed packets[6].  AODV discoveries are only initiated when needed and so additional network traffic is not present.  This feature also presents a limitation causing high latency when changing network conditions cause routes to be invalidated and rediscovered. It also affects network throughput by limiting the amount of packets that can be sent when routes are not known.

Digi International, the company behind DigiMesh claims that networks with around 500 nodes are possible and within the limitations of the standard.  Due to our limited network traffic needs this limit should not be a concern.  Any latency introduced by route discoveries on large networks would not be a concern because the time constraint for our product’s users is measured in minutes, and the network delay introduced being measured in hundreds of milliseconds.

The power consumption of the Digi modules can be reduced by configuring the routing nodes to be sleeping.  This setting is configurable and each device will be synchronized to a automatically elected time coordinating node.  With our network measuring user response time in minutes, a sleeping schedule of 1/20 or 1/30 seconds would not likely cause problems.  Sleeping for these time scales would increase runtime by a nonlinear inverse of the fraction. Our estimates place a 1/20 seconds schedule at decreasing power by 57\%.

The DigiMesh protocol, while proprietary, meets all the requirements for our client device communication.  With claimed scalability to 500+ nodes it will provide the scalability we intend to support.  The 900 MHz frequency range is expected to provide the needed coverage range and allow for a network of connected routing-capable and self-healing end nodes to provide maximum coverage  by allowing distant end nodes to connect while within range of at least a single node.

\subsection{Button}

\subsubsection{Functional Considerations}

The main design requirement of the button is fairly simple: to allow a user to press a button which will transmit a distress signal to a paired base station.  However, this signal should only be sent when the button is deliberately pressed or an accelerometer reads dangerous levels, and the button should have functionality to cancel this distress call fairly easily.  Aside from user input, distinct feedback is important in user interactive products and the button should relay useful information to/from the wearer via LED indicators, vibration, and/or other forms of low-power notification.  Price is also an important factor involved in the design of this button, and while peripherals like USB connectivity and more sophisticated energy storage solutions could improve user ease-of-use, additional features will be omitted until later prototype models.  The form factor must also be as small as possible, putting limits on battery, board, and antenna sizes.  As the brains of the button, we hunted for a low-powered, Bluetooth functional microcontroller to meet these functional requirements.

\subsubsection{Microcontroller}

Due to our desires to use Bluetooth Low Energy (BLE) and to accurately communicate ‘HELP’/’CANCEL’ signals between button and base-station, we selected a microcontroller which would optimize the size, cost, and power efficiency.

\begin{table*}[t]
  \centering
  \begin{tabular}{>{\bfseries}l|l l l}
    Device & ST BlueNRG-1 & Kinetis KW30Z & Atmel SAMB11 \\
    \hline
    Datasheet & Link & Link & Link \\
    DevBoard Price & \$111.25 & \$145 for 2 & \\
    Chip Price & \$3.37 with 10 & \$4.72 with 50 & \$3.51 \\
    2.4 GHz Antenna & No & Yes (Bluetooth, Zigbee) & No \\
    Flash & 160 kB & 128-512 kB & 256 kB \\
    RAM & 24 kB & 64 kB & 128 kB \\
    Supply & 1.7-3.6 V & 3.6 V & 2.3-3.6 V \\
    Power Consumption (avg) & 0.174 mWh & 0.596 mWh & 0.544 mWh
  \end{tabular}
  \caption{Button Microcontroller Costs and Specifications}
\end{table*}

TODO: Missing figure

The values given for average power consumption are based on a protocol initially agreed upon by our group, and average powers are presented as rough estimates for the power used by the button as a whole.  The maximum energy stored in a standard 3V coin cell battery is 1000mAh, which constrains us to a power budget of 3000 mWh/battery at most.  While these microcontrollers have similar characteristics of operation, the most power and cost efficient solution is the ST BlueNRG-1, so we decided to explore further button development with it.

\section{General Specifications}

\subsection{Evaluation of Available Components}

\subsubsection{Button Microcontroller}

\textbf {Hardware:} STMicroelectronics BlueNRG-1

\begin{itemize}
  \item Core and Memory
    \begin{itemize}
      \item 16 or 32 MHz ARM Cortex-M0
      \item 160 KB Flash
      \item 24 KB RAM with retention (two 12 KB
        \item banks)
    \end{itemize}
  \item Bluetooth Low Energy Radio Features
    \begin{itemize}
      \item Bluetooth specification compliant master,
      \item slave and multiple roles simultaneously,
      \item single-mode Bluetooth low energy system-on-chip
      \item Battery voltage monitor and temperature sensor
    \end{itemize}
  \item Power
    \begin{itemize}
      \item Operating supply voltage: from 1.7 to 3.6 V
      \item Integrated linear regulator and DC-DC stepdown converter
      \item Operating temperature range: -40 °C to 105°C
      \item Down to 1µuA current consumption with active BLE stack (sleep mode)
    \end{itemize}
  \item Interfaces
    \begin{itemize}
      \item 1 x UART interface
      \item 1 x SPI; 2 x I2C interface
      \item 14 or 15 GPIO
      \item 2 x multifunction timer
      \item 10-bit ADC
      \item Watchdog and RTC
      \item DMA controller
      \item PDM stream processor
    \end{itemize}
\end{itemize}

\subsubsection{Base Station Microcontroller}

\textbf {Hardware:} Atmel SAMB11

\begin{itemize}
  \item Core and Memory
    \begin{itemize}
      \item 26 MHz ARM® Cortex®-M0+ core
      \item 256 KB Stacked Flash memory
      \item 128 KB Embedded ROM
        128 KB Embedded RAM
    \end{itemize}
  \item Low Power Consumption and Operating Voltage Ranges
    \begin{itemize}
      \item Nine low-power modes to provide power optimization based on application requirements
      \item Typical Rx/Tx Current (DC/DC Enabled): 6.8/6.1 mA
      \item $1.1\mu A$ sleep current (8K RAM retention and RTC running)
      \item 3.0mA peak TX current (0dBm, 3.6V)
      \item 4.2mA peak RX current (3.6V, -93dBm sensitivity)
      \item Bypass Voltage: 1.71V to 3.6V
      \item DCDC Converter Buck Configuration: 2.1V to 4.2V
      \item DCDC Converter Boost Configuration: 0.9V to 1.795V
      \item Analog modules
      \item 11-bit 4 Channel Analog-to-Digital (ADC)
      \item 6-bit High Speed Analog Comparator (CMP)
    \end{itemize}
  \item Bluetooth Low Energy Radio Features
    \begin{itemize}
      \item 2.4 GHz Bluetooth Low Energy version 4.2 Compliant
      \item Typical Receiver Sensitivity (BLE) = -95 dBm
      \item Programmable Transmitter Output Power up to +3.5 dBm
    \end{itemize}
\end{itemize}

\subsubsection{Base Station Interconnect}

\textbf {Hardware:} Digi ZigBee-PRO 900HP RF Module

\begin{itemize}
  \item Transceiver Info
    \begin{itemize}
      \item North American ISM band from 902 to 928 MHz
      \item 3.3 V UART communication
      \item DigiMesh Mesh Network protocol with self-healing, auto-configuration of nodes, and synchronized sleep.
    \end{itemize}
  \item Spectrum and Data Rates
    \begin{itemize}
      \item RF Data Rate  10 Kbps or 200 Kbps
      \item Indoor/Urban Range  10 Kbps: up to 2000 ft (610 m); 200 Kbps: up to 1000 ft (305 m)
      \item Outdoor/ Line-Of-Sight Range  10 Kbps: up to 9 miles (14 km); 200 Kbps: up to 4 miles (6.5 km) ( w/ 2.1 dB dipole antennas)
      \item Transmit Power up to 24 dBm (250 mW) software selectable
      \item Receiver Sensitivity  -101 dBm @ 200 Kbps, -110 dBm @ 10 Kbps
    \end{itemize}
  \item Current Consumption
    \begin{itemize}
      \item SLEEP = 2.5 µA
      \item IDLE/RX\_ON = 29 mA typical at 3.3 V (35 mA max)
      \item BUSY\_TX = 215 mA typical for +24 dBm, 250 mW (290 mA max)
    \end{itemize}
\end{itemize}

\subsubsection{Display}

Crystalfontz CFAH1604A-TMI-JT Character LCD Module

16x4 Parallel Character LCD with backlight

\subsubsection{Buttons}
TI LM8330 Keypad IC

\subsubsection{Power Budget}

\begin{table*}[t]
  \centering
  \begin{tabular}{>{\bfseries}l|l l l l l}
    Device & $V_{CC} (V)$ & $I_{CC}$ (mA) & $P$ (mW) & Duty Cycle & Energy (mWh) \\
    \hline
    Atmel SAMB11 uC only & 3.3 & 0.85 & 2.805 & 0.6 & 1.683 \\
    Atmel SAMB11 RX & 3.3 & 4.5 & 14.85 & 0.2 & 2.97 \\
    Atmel SAMB11 TX & 3.3 & 3 & 9.9 & 0.2 & 1.98 \\
    Atmel SAMB11 Standby & 3.3 & 0.00125 & 0.004125 & 0 & 0 \\
    XBee-Pro 900HP 900MHz Antenna RX & 3.3 & 29 & 95.7 & 0.1 & 9.57 \\
    XBee-Pro 900HP 900MHz Antenna TX & 3.3 & 120 & 396 & 0.1 & 39.6 \\
    XBee-Pro 900HP 900MHz Antenna Standby & 3.3 & 0.0025 & 0.00825 & 0.8 & 0.0066 \\
    TI LM8330 Keypad & 3.3 & 0.03 & 0.054 & 0.05 & 0.0027 \\
    Crystalfontz CFAH1604A-YYH-JT LCD Controller & 3.3 & 1.2 & 3.96 & 0.05 & 0.198 \\
    Crystalfontz CFAH1604A-YYH-JT LCD Backlight & 5 & 220 & 1100 & 0.05 & 46.2 \\
    Audio Speaker & 12 & 16 & 192 & 0.01 & 1.92 \\
    \hline
    Max & & 394.58375 & & & \\
    Total & & & 1815.281375 & & 112.9303
  \end{tabular}
  \caption{Base Station Power Budget}
\end{table*}

This power budget is derived for a typical use case of the base station.  Each component is detailed with its individual power consumption, and in combination with the estimated on-time in an hour’s time, we estimate the total energy consumed by the system in one hour.  The peripherals needed for user interface can be seen as the most power hungry elements on the base station.  However, since our system is designed as a safety product that should hopefully go off no more than once every 6 months, the audio speaker should contribute minimally to the power usage in the grand scheme of things.  The LCD display will most likely be used more often then other peripherals, as it will display more useful information, such as status updates and neighbor information.  Beyond the initial setup, the LCD should also not be in use for most of the time.

Given the on-time percentage depicted in the table, the system uses about 112.93 mWh of energy in one hour. Assuming a 700 mAh rechargeable NiMH backup battery, the base station can last about 23 hours without wall power. Because our percentages are generously estimated and software functionality has not yet been optimized for maximum power savings, an even better battery life could be achieved in future revisions.

\begin{table*}[t]
  \centering
  \begin{tabular}{>{\bfseries}l|l l l l l}
    Device & $V_{CC} (V)$ & $I_{CC}$ (mA) & $P$ (mW) & Duty Cycle & Energy (mWh) \\
    \hline
    2.4GHz BTLE Tx & 3 & 15.1 & 45.3 & 0.00028 & 0.0127 \\
    2.4GHz BTLE Rx & 3 & 7.7 & 23.1 & 0.00028 & 0.00647 \\
    GPIO Driving LEDs [x2] & 3 & 4.4 & 13.2 & 0.00056 & 0.00739 \\
    CPU, Flash, and RAM & 3 & 1.9 & 5.7 & 0.00056 & 0.00319 \\
    24KB RAM retention \& LFO RO ON & 3 & 0.0021 & 0.0063 & 0.994 & 0.00626 \\
    Peripheral GPIO [x2] (for push-button and accelerometer) & 3 & 0.04 & 0.12 & 1.00 & 0.12 \\
    ST Accelerometer (LIS2DE12) & 3 & 0.006 & 0.018 & 1.00 & 0.018 \\
    \hline
    Max & & 21.4 & & & \\
    Total & & & & & 0.174
  \end{tabular}
  \caption{Button Power Budget}
\end{table*}

Our button power budget gives us a quantitative tool to better understand its energy storage requirements.  The values above are based off of theoretically agreed-upon (by the group) and reasonable estimates of a typical user case.

\subsubsection{Trade Studies}
Trade studies were conducted as a tool for the decision-making process for both the button and base station microcontrollers.  Our research was assimilated, evaluation criteria developed, and the different options were then graded. The evaluation criteria for the button microcontroller were cost, power, transmission range, integrability, and schedule.  The evaluation criteria for the base station microcontroller were cost, memory, functionality, integrability, and schedule.  Each criteria was assigned a weighted percentage according to its significance to the project using a grading scale of five, with the lowest score representing the optimum choice.

\begin{figure}[ht] 	% Several different modifiers can be used
\centering
\includegraphics[width=0.4\textwidth]{ButtonTS.png}
\caption{ \space Button Trade Study Results}
\label{HaHa Button TS}
\end{figure}

\begin{figure}[ht] 	% Several different modifiers can be used
\centering
\includegraphics[width=0.4\textwidth]{BaseTS.png}
\caption{ \space Base Station Trade Study Results}
\label{HaHa Base TS}
\end{figure}

\section{System Overview}

\subsection{Product Perspective}
The base station and button will be separate hardware devices which facilitate the communication of a “help request” when a user presses the button on either the base station or the remote button. The base station will contain a priority list of contacts who will be contacted in an order set by the user. Alerted users will respond to the help request through use of their own base station.  If a user does not respond to a help request, the next person on the list is contacted. If none of the people on the help list respond, any station in the nearby vicinity is contacted.

\subsubsection{Design Method}
The software design portion of the product will utilize the C-programming language.

\subsubsection{User Interfaces}
The base station is equipped with an LCD display, which will facilitate user interaction with the device, in addition to showing critical information during emergency situations. Configuration of the device shall be done using a series of menus displayed on the LCD display. The exception to this are several dedicated switches/buttons that control the contrast of the display, reset device power, and trigger the help request functionality. The rest of the device functionality shall be accessed through the menu system, and facilitating this is a set of four directional buttons, as shown in Figure \ref{dev buttons}. These buttons shall have the following functionality:

\begin{LaTeXdescription}
	\item[Up] Scroll up in the list of choices
    \item[Down] Scroll down in the list of choices
    \item[Left] Move back/cancel, exiting a submenu
    \item[Right] Select, selecting the current item (that which is listed at the top of the screeu) or open the corresponding submenu
\end{LaTeXdescription}

\begin{figure}
	\centering
	\includegraphics[width=0.48\textwidth]{dev_buttons.jpg}
    \caption{Directional Buttons on the Development Board}
    \label{dev buttons}
\end{figure}

The base station also allows an external device (such as a lamp) to be plugged directly into it. This device will be toggled on and off during emergency situations.

The button is equipped with a single button to trigger the emergency response system. Different inputs (such as a single press, three successive presses, and a hold) serve different functionality. It also has LED indicators for periodic synchronization pulses, confirmations that the help request has been received, and to confirm another user has answered the distress call.

\subsubsection{Hardware Interfaces}
The hardware of the base station and the remote button will be designed and developed by the team.

\subsubsection{Operations}
The user of the base station will be required to setup a priority contact list.  The system will have the option of scanning for all available network nodes which will display the name of nearby users and their node number.  Configuring of a user to contact can also be done via a pre-shared ID number which each base station user can obtain through the user interface.

When a help request is sent, the initiating user will be able to cancel the request by long-pressing the help button either on the base station or remote button.

\subsection{Product Functions}

\begin{LaTeXdescription}
\item[Send Help Request] This function will send a help request beginning with the first base station in the user’s contact list. Functionality for this will be present on both the remote button and the base station.
\item[Cancel Help Request] This function cancels a help request alert. In the event a user has already responded to a help request, an additional signal will be sent to the responder’s box informing them of the cancelled request.
\item[Setup Contact List] This set of functions will facilitate configuring the user’s contact list.  Users can either enter ID numbers of the neighbors they wish to add or they can scan the network for them.
\item[Scan Network] This function scans the network for available users and provides user contact details. It will interface directly with the Setup Contact List.
\end{LaTeXdescription}

\subsubsection{Software}

\subsubsection{Application Networking Protocols}

This section is an overview of the networking protocols which must be determined.  Specific information such as packet design will be fleshed out in the next section, System Architecture.

\begin{itemize}
  \item Button to Box
    \begin{itemize}
      \item Base station reads button battery level using Bluetooth API. (performed frequently)
      \item Button activates “Help Alert” state on the base station. (button pressed for 3 sec)
      \item Button activates “Immediate Help Alert” state on the base station. (button pressed for 6 sec)
    \end{itemize}
  \item Base Station to Base Station
    \begin{itemize}
      \item Discovering neighbors. (broadcast)
      \item Sends help request to specific and possibly distant neighbor. (ACK’d)
      \item Sends help alert received signal back to requester. (broadcast)
      \item Sends help alert resolved signal to all neighbors requested. (broadcast)
    \end{itemize}
\end{itemize}

\subsection{System Diagrams}

\subsubsection{System Block Diagram}

TODO: Missing figure

\subsection{Network Simulation}

NS3 is a discrete-event network simulator. A discrete-event are those events which occur in a sequence over time. Each event occurs at a particular instant in time and marks a change of state in the system. NS3 facilitates the defining of this system and mechanisms to view and record the state of every mechanism within the system.\\

Our goal with NS3 is to simulate the network as closely as possible to the DigiMesh protocol using the AODV algorithm as a basis. Packet broadcasts that approximate our own application layer packets will be configured and the number of nodes scaled such that the network performance can be analyzed as the number of nodes or amount of network traffic increases. This will allow us to realize with some level of certainty the level at which our network will be deployable. It also will provide us with a basis for any future troubleshooting of the network that we might encounter a need for.\\

In its current form, the simulator creates a configurable number of nodes a random x-y-z distance apart and has the two furthest nodes communicate packets. The communications regarding the AODV algorithm are traced and logged to the screen along with the communication packets. As state above, this will be modified in the coming quarter to be more representative of our system.\\

Below is an example output:

\begin{lstlisting}
Creating 10 nodes 100 m apart.
x=80.7772, y=45.6102, z=0.458782
x=48.6495, y=72.3938, z=0.448497
x=22.2484, y=53.5658, z=0.513956
x=26.1946, y=19.7807, z=0.446207
x=68.3362, y=54.6383, z=0.830643
x=92.0653, y=28.9132, z=0.773134
x=54.4156, y=93.372, z=0.662756
x=14.7075, y=19.3661, z=0.662903
x=46.9031, y=68.3726, z=0.521647
x=53.8059, y=83.5943, z=0.0441295
Starting simulation for 10 s ...
[node 0] Starting at time 36ms
[node 1] Starting at time 32ms
[node 2] Starting at time 73ms
[node 3] Starting at time 53ms
[node 4] Starting at time 95ms
[node 5] Starting at time 95ms
[node 6] Starting at time 99ms
[node 7] Starting at time 22ms
[node 8] Starting at time 11ms
[node 9] Starting at time 27ms
PING  10.0.0.10 56(84) bytes of data.
[node 0] Valid Route not found
[node 0] Send RREQ with id 1 to socket
[node 6] Starting at time 99ms
[node 7] Starting at time 22ms
[node 8] Starting at time 11ms
[node 9] Starting at time 27ms
PING  10.0.0.10 56(84) bytes of data.
[node 0] Valid Route not found
[node 0] Send RREQ with id 1 to socket
on 10.0.0.10
[node 9] AODV node 0x1bf5760 received 
a AODV packet from 10.0.0.1 to 10.0.0.10
[node 9] Send reply since I am the 
destination
[node 9] Exist route to 10.0.0.1 from 
interface 10.0.0.10
[node 9] Updating VALID route
[node 9] Updating VALID route
\end{lstlisting}

\section{System Architecture}

\subsection{Software}

The button and base station are paired together to form a master/slave relationship where the button is the slave and the base station the master.  The button transmits signals to the base station in the form of packets.  After receiving a packet, the base station processes the information accordingly to either forward the message to other base stations or back to the button.  The button is usually in sleep mode and only wakes up if a button event occurs.  The following cases are envisioned requirements to be fulfilled by the respective hardware component’s libraries.

\subsubsection{Button}

\begin{LaTeXdescription}
\item[Case 1]: Initial connection established between button and base station.
  \begin{itemize}
    \item The button device is initially in standby mode waiting to be paired.  The button is pressed on the base station to establish the initial connection process.  The button on the button device is then pressed and held for a several seconds until a message appears on the base station confirming successful connection.  This initial process only occurs once when a single button device is registered to a base station.
  \end{itemize}
\item[Case 2]: The button is pressed within range of the base station.
  \begin{itemize}
    \item The button can either send a help alert signal or cancel a help alert packet.
    \item If the button is pressed and held down for 3 seconds, it will transmit a help alert packet to the base station.  After acknowledging a help alert packet was received, the base station will transmit the help alert to the contacts listed in the requester’s network list.  If the first person on the list does not respond within a certain timeframe, it will timeout and go to the next contact on the list.  If no one on the list responds, then the packet is sent to additional neighbors.  If none of these additional neighbors respond, the packet is then sent through a certain number of network hops which will be calculated according to the density of base stations existing base stations.  If no one still responds, then the packet is sent through the entire network of base stations as a final pass.
    \item If the button is pressed 3 times consecutively, then it will transmit a cancellation of the help request packet.  After receiving and acknowledging this type of packet was indeed received, it will cancel the help request.  There will be a help request cancellation message displayed on the base station.
  \end{itemize}
\item[Case 3]: The button is pressed when out of range of the base station.
  \begin{itemize}
    \item If the button is out of range of the base station, it will disconnect and won’t be able to transmit or receive any packets.  Once it is back within range of the base station, it will automatically connect to it and a successful connection message will be displayed on the base station.
  \end{itemize}
\item[Case 4]:  The button is not pressed but wakes up to send notifications to the base stations within range.
  \begin{itemize}
    \item The base station continually polls information from the button.  The percentage of the button’s battery life is polled by the base station for a given amount of time.  Another query occurs every 6 hours to determine if the button is still connected to the base station.  If base station does not receive a signal from the button within 12 hours, it will notify the user there is an issue with the button by displaying this information on the base station.
    \item When the battery is drained under a certain percentage, it would notify the user the battery is low.  A message will be displayed on the base station informing the user to replace the button battery.  The LED on the button will also become a solid red color to signal the low battery life and it will eventually begin blinking if the battery life is critically low and in need of immediate replacement.
  \end{itemize}
\item[Case 5]: The button is not pressed but wakes up to send notifications to base stations not within range.
  \begin{itemize}
    \item Same conditions as Case 3.
  \end{itemize}
\end{LaTeXdescription}

\subsubsection{Base Station}

\begin{figure}[ht] 	% There are several different modifiers that can be used in [].
\centering
\includegraphics[width=0.4\textwidth]{hahamap.png}
\caption{ \space Software map.}
\label{HaHa Application}
\end{figure}

\textbf {Application Modules}

As can be seen by the UML layout in Fig. \ref{HaHa Application}, our application will be a series of modules meant to interconnect and provide the needed functionality. By creating modules for each portion of software it will be possible for modifications to be easily implemented by replacing the relevant modules. This project is being developed as an open-source project and as such it is our endeavor to allow modifying the system easy and straightforward.

Chief among this modularity is the “network device” layer we are developing to utilize a Digi ZigBee module. This layer and its modules will be responsible for receiving data in the form of a character string from the upper application and converting it as necessary to be sent as an XBee network frame. On the receiving end it will do the reverse operation and provide the upper layer with a character string. In the event that a different transceiver wished to be used, creating and implementing similar functions would allow the majority of the application layer to remain unmodified.

\textbf{Display}

The base station was equipped with a Crystalfontz CFAH1604A-TMI-JT 16x4 LCD display. A 4 line display was a requirement for the project, as the device would be used during high stress situations, in which users would want to see all necessary information on one screen without having to interact with the device interface. The Crystalfontz display was the cheapest 4 line display available find on the market, and had the added benefit of being based off of the Hitachi HD44780U, a widely used and well documented component.

Similar to other software components in the project, we abstracted the display drivers into two separate modules, a device specific component and a generic component:

\begin{LaTeXdescription}
	\item[lcd\_device] holds device specific code and macros, such as the display size (\lstinline[columns=fixed]{LCD_ROWS=4, LCD_COLS=16}), and facilitates the communication with the physical device, which in this case means writing to specific registers
    \item[lcd] abstracts the physical device, by buffering the output to the LCD screen with arrays implemented in software. A single \lstinline[columns=fixed]{lcd_update} function updates the entire LCD screen contents.
\end{LaTeXdescription}

There were some complications with implementing the LCD driver. First is the subtle differences between the Crystalfontz display and the Hitachi specification. After the initial version of the driver was implemented, we noticed that the display would only work correctly if the delay in between commands was set to more than 90 ms. This was unusual, as the datasheet only required a delay of around 40 ms in between commands during the initialization phase, and then 3 ms when writing to the screen. We eventually discovered that this was due to the Crystalfontz controller expecting a specific initialization command to be sent twice, which was only briefly mentioned on their datasheet and was not part of the Hitachi specification at all. This forced us to use the Crystalfontz datasheet for the rest of development, which was overall not as intuitively laid out as the one given by Hitachi. 

Another design decision in integrating the LCD display was to use 4 bit mode, as opposed to 8 bit mode. 8 bit mode is standard, in which each character is specified in 8 bits, which is standard ASCII. 4 bit mode sets the controller to read in only 4 bits of data at a time, thus requiring two register writes per character/command to 8 bit mode's 1 write. The justification for doing so was to free up pins on the microcontroller, which had many components to interface to. In addition to the 4 data pins, the LCD display already required 3 pins for control, making it the largest component in terms of pin usage. However, this had some drawbacks, in both development and performance. The Crystalfontz datasheet had minimal examples for operating in 4 bit mode, which meant that each 8 bit command had to be "translated" into 4 bit commands, which was tedious and hindered the progress of debugging. On the performance side, since each command took twice as many writes, when compounded with the mandatory delay between commands, this produced noticeable lag when communicating 

Besides the obvious benefit of being able to swap out the device specific code as necessary, there was a very specific reason the \lstinline[columns=fixed]{lcd} module needed to buffer the LCD output, having to do with the way the LCD controller operated specifically for a 16x4 display. In a standard 16x2 display, the controller would read in 16 characters corresponding to the top line of the display. It would then expect a command to change the write address to the second row, and take in an additional 16 characters. However, for a 16x4 display, the same controller was adapted to work in a similar way. In order to accomplish this, the third and fourth rows are simply extensions of the first and second rows, resulting in the following unusual behavior: writing 16 characters for the first row, then 16 characters for the third row, followed by the address command and following 16 characters for the second then fourth rows respectively. What this means in practice is that the LCD display cannot support writing characters in order, which makes buffering necessary in order to implement functions such as \lstinline[columns=fixed]{lcd_set_line()}.

\textbf{User Interface}

The user interface of the base station is composed of a tree of menu items, a data structure defined in the \lstinline[columns=fixed]{menu} software module (included in Appendix \ref{MenuItem Code}). Each menu item has reference to it's parent, it's next and previous sibling, and it's first child, all of which are also menu items. These four references relate to the four cardinal directions given in the user interface:

\begin{LaTeXdescription}
	\item[Up] Scrolls up in the current list of menus, setting the cursor as the previous sibling of the current item (or if current is the first sibling, to it's last sibling), and then calling it's \lstinline[columns=fixed]{onView()} method.
    \item[Down] Scrolls down in the current list of menus, setting the cursor as the next sibling of the current item (or if current is the last sibling, to it's first sibling), and then calling it's \lstinline[columns=fixed]{onView()} method.
    \item[Left] Moves back (up) in the menu tree, setting the current item as the parent and then calling it's \lstinline[columns=fixed]{onView()} method.
    \item[Right] Selects the current item, calling it's \lstinline[columns=fixed]{onClick()} method. By default, this sets the cursor to the child of the current item, if it exists.
\end{LaTeXdescription}

Referenced above, each menu item also has two function callbacks assigned to it, which are referred to as the \lstinline[columns=fixed]{onView()} and \lstinline[columns=fixed]{onClick()} methods. These functions are called throughout the menu navigation process, and are what allow the menu to dynamically alter both itself and the other software modules of the base station. 

The \lstinline[columns=fixed]{onView()} method is called whenever a menu is viewed. By default, this takes form of printing the current menu item's string \lstinline[columns=fixed]{value} on the top line of the LCD screen, with it's next siblings shown on the following lines, as seen in Figure \ref{menu view default}. This gives the user context as to what they are currently selecting, and what is available to be selected should the user choose to move up/down. However, there are some cases where this is not ideal, such as in the User Info submenu. Because some fields of user information typically span more than one line (the address field specifically is a good example of this), each child of the User Info menu item has a custom \lstinline[columns=fixed]{onView()} callback, which takes up the entire screen. Pressing up and down for this submenu still navigates between items, but overwrites the entire screen, like shown in Figure \ref{menu view user info}. In addition, the second line of the LCD screen is populated with user info taken from a datastructure defined in another software module. Another custom \lstinline[columns=fixed]{onView()} method is used in the text input feature, which will be discussed in greater detail later.

\begin{figure}
  \begin{subfigure}{0.48\textwidth}
  	\centering
    \LCD{4}{16} |Friends List   >|
                |User Info       |
                |Device Settings |
                |                |
    \caption{Default View}
    \label{menu view default}
  \end{subfigure}
  \begin{subfigure}{0.48\textwidth}
    \centering
    \LCD{4}{16} |USER ADDRESS:   |
                |1157 High St. Sa|
                |nta Cruz, CA    |
                |vMore      Edit>|
    \caption{User Info Custom View}
    \label{menu view user info}
  \end{subfigure}
  \caption{\lstinline[columns=fixed]{onView()} methods}
\end{figure}

As mentioned in the table, by default, the \lstinline[columns=fixed]{onClick()} method of a menu item should navigate to it's corresponding submenu (it's first child). However, there are some situations where it is necessary to do more. For example, every time the Friends List menu item is clicked, it's children are regenerated based on the current state of the software friend list. The end result is the same (the cursor being set as the first friend in the list, which is the first child of the Friends List menu item), but because other actions are going on in the background to produce these children, the \lstinline[columns=fixed]{onClick()} method had to be overwritten. Another good example is again the User Info items mentioned previously. Selecting them starts the text input feature, which creates a large subtree below the current item.

The Help Request and Friend Request features use a combination of the \lstinline[columns=fixed]{onView()} and \lstinline[columns=fixed]{onClick()} methods to provide a cohesive message. When a request is received over the network, the cursor is forcibly set to the menu item Help Response, with the view method producing a screen like that shown on Figure \ref{menu view help}. This method, similar to the User Info menu item, takes in a name (this time from a network packet) and displays it on the screen. The resultant menu tree after this process is completed looks like this:

\begin{itemize}
	\item Help Response Root
    \begin{LaTeXdescription}
        \item[child] Help Response Deny
        	\begin{LaTeXdescription}
                \item[onClick] NULL
                \item[onView] Send "Cannot help you" packet, set current to root
                \item[child] Help Response
                    \begin{LaTeXdescription}
                        \item[onClick] Send "Help is on the way" packet, set current to root
                        \item[onView] Show information on the person who needs help (see Figure \ref{menu view help})
    				\end{LaTeXdescription}
            \end{LaTeXdescription}
    \end{LaTeXdescription}
\end{itemize}

\begin{figure}
  \begin{subfigure}{0.48\textwidth}
  \centering
  \LCD{4}{16} |! HELP REQUEST !|
              |Donald Wiberg   |
              |                |
              |<BACK   RESPOND>|
  \caption{Help Request Text}
  \end{subfigure}
  \begin{subfigure}{0.48\textwidth}
  	\centering
  	\includegraphics[width=\textwidth]{device_help_request.jpg}
    \caption{Help Request on Physical Device}
  \end{subfigure}
  \caption{Help Request View}
  \label{menu view help}
\end{figure}

Thus, when the user presses right, the click method of Help Response is called, and the affirmative packet is sent. Likewise, if the user presses left, then the menu transitions to the Help Response Deny (which is the parent of the Help Response menu item), and it's view method is called. This causes the negative packet to be sent. In either case, the user is immediately transitioned back to the root menu item, which is the default startup screen.

\textbf{Text Input}

The base station facilitates text input much the same way as a networked printer or old school arcade game would. Up (Figure \ref{menu input up}) and down (Figure \ref{menu input down}) changes the current letter at the cursor, left (Figure \ref{menu input left}) deletes a character, and right (Figure \ref{menu input right}) selects the current character displayed. Depending on the context of the text input, the characters available may differ. For instance, names support upper case and lower case letters, in addition to spaces, which when scrolling through the list of letters is represented by an underscore. Likewise, addresses support only upper case letters to limit complexity, but also include spaces, numbers, and periods. Lastly, phone numbers only support numbers. In all cases, when scrolling through the list of characters, there is the addition of a special character, the right arrow ($\rightarrow$). This represents the enter key, and signifies that the user is finished with the string they are editing and would like to save. Deleting the entire string, and then continuing to press left exits the editor, discarding changes.

\begin{figure}
  \begin{subfigure}{0.48\textwidth}
  	\centering
    \LCD{4}{16} |Editing Name:  >|
                |STEPHEN PETERS  |
                |                |
                |                |
    \caption{Initial State}
    \label{menu input start}
  \end{subfigure}
  \begin{subfigure}{0.48\textwidth}
  	\centering
    \LCD{4}{16} |Editing Name:  >|
                |STEPHEN PETERR  |
                |                |
                |                |
    \caption{Up Press (prev letter)}
    \label{menu input up}
  \end{subfigure}
  \begin{subfigure}{0.48\textwidth}
  	\centering
    \LCD{4}{16} |Editing Name:  >|
                |STEPHEN PETERT  |
                |                |
                |                |
    \caption{Down Press (next letter)}
    \label{menu input down}
  \end{subfigure}
  \begin{subfigure}{0.48\textwidth}
  	\centering
    \LCD{4}{16} |Editing Name:  >|
                |STEPHEN PETE_   |
                |                |
                |                |
    \caption{Left Press (delete letter)}
    \label{menu input left}
  \end{subfigure}
  \begin{subfigure}{0.48\textwidth}
  	\centering
    \LCD{4}{16} |Editing Name:  >|
                |STEPHEN PETERS_ |
                |                |
                |                |
    \caption{Right Press (confim letter)}
    \label{menu input right}
  \end{subfigure}
  \caption{Text Input States}
\end{figure}
\begin{comment}

\textbf{User Interface}

The base station design was motivated by two primary goals: (1) to be easy and intuitive for elderly people to operate in emergency situations, and (2) to be composed of inexpensive components to support wide adoption by the community.  These goals were somewhat contradictory: inexpensive components are often the most difficult to work with, while better components are more expensive which will drive up the overall cost of the system.  Our sponsor, Professor Emeritus Donald Wiberg, gave us a soft limit of approximately \$50 for the entire system’s components (not including packaging), which greatly limited which components were feasible for our project.

Thus, it was our decision to make the base station’s interface as easy to use as possible, which made it critical that the user interface software be as simple as possible. This requires minimizing the amount of input the user can provide, to make it immediately apparent what is expected of them in order to perform a certain action.  This has the additional effect of minimizing our state machine, which should make programming the device much easier.

The base station is equipped with 5 buttons, which serve the following purposes:

\begin{LaTeXdescription}
\item[Up] Move up in the list of menu items
\item[Down] Move down in a list of menu items
\item[Left/Back/Cancel] Cancel action and move back in the menu tree
\item[Right/Forward/Select] Select the current menu item (the one displayed the top of the screen), and perform said action/move into said submenu
\item[Help Request] Trigger the emergency request system, performing the same functionality as pressing your Bluetooth button
\end{LaTeXdescription}

The LCD screen displays a list of menu items for the current view, in standard operation.  These menu items can be thought of as (possibly selectable) options which the user can place their cursor on.  In the system, the cursor, or “currently selected item”, is the topmost item in the list.  Below the cursor is a variable length list of other viable options which can be scrolled through using the up and down buttons.  Moving your cursor shifts the entire list up or down one space and the lists have currently been configured to wrap around at the end, moving back to the top upon reaching the bottom (but can be easily configured to not do so).  If the cursor is selectable, then the last character of the line is the “>” character, to signify a right button press will select that item. Selectable items that are not currently the cursor have the arrow “→” character as their last character to indicate as such. The back arrow will navigate to the current item’s parent and exit the existing submenu.
\end{comment}

\subsection{Network Application Layer}

The Application Layer between base station communications has several different commands which can be transmitted.  Most of these events are triggered by the user but since multiple users can contact any station at any time, we need a robust way to handle each request. 
Each station will have a packet queue where it can send packets and expect to receive packets from the destinations in which they were sent.  These queues will have timeouts to protect the integrity of the device as well as allowing communication between devices.

\subsubsection{Network Adaptation Layer}

The network adaptation layer is a module that isolates the HaHa application from the network being used to transmit data between the base stations. 

\begin{figure}[ht] 	% There are several different modifiers that can be used in [].
\centering
\includegraphics[width=0.4\textwidth]{NetAdaptation.png}
\caption{ \space Network adaptation layer.}
\label{Network Adaptation}
\end{figure}

By implementing the externally accessible functions contained in the {\it networkdevice} files, the network communication device and protocols can be modified or replaced. The functions for a new device need only accept the appropriate parameters and return the expected data types in order to effectively change the any aspect of the communication device or the network medium or protocol.

This modular design approach was utilized during the development of this project. It became necessary to implement a logging feature for use in a concept deployment study. Because the network functionality was completely separate from the rest of the application, the implementation of a {\it catch-all} mechanism was efficiently implemented and without change to the core HaHa application, except for a modified receiver running a modified code-base to store the data.

\subsubsection{Packet Structure}

\begin{sidewaystable*}[t]
%\begin{table*}[t]
  \centering
  \begin{tabular}{>{\bfseries}l|l l l}
    Name & OPCODE (1 byte) & Flags (1 Byte, 8 flags) & Data (2-98 Bytes) \\
    \hline
    PING\_REQUEST & 0x01 & ACK & DESTUID \\
    ACK PING\_REQUEST & PING\_REQUEST & ACK & SRCUID, SRCNAME \\
    HELP\_REQUEST & 0x02 & ACK|CANCEL|IMM & SRCUID, HOME, PHONE \\
    ACK HELP\_REQUEST & HELP\_REQUEST & ACK|CANCEL|IMM & SRCUID \\
    HELP\_RESPONSE & 0x03 & ACK|ACCEPT & SRCUID \\
    ACK HELP\_RESPONSE & HELP\_RESPONSE & ACK|ACCEPT & SRCUID \\
    HELP\_FROM\_ANYONE (BRDCST) & 0x04 & CANCEL|IMM & SRCUID, TTL, SRCNAME \\
    HELP\_FROM\_ANYONE\_RESP & 0x05 & ACK|ACCEPT & SRCUID, DESTUID, SRCNAME \\
    ACK HELP\_FROM\_ANYONE\_RESP & HELP\_FROM\_ANYONE\_RESP & ACK & SUID, DUID, HOME, PHONE \\
    FIND\_HOPS\_REQUEST (BRDCST) & 0x06 & ORIGINUID \\
    FIND\_HOPS\_RESPONSE & 0x07 & ACK & SRCUID, ORIGINUID \\
    ACK FIND\_HOPS\_RESP & FIND\_HOPS\_RESPONSE & ACK & SRCUID, DESTUID \\
    FIND\_NEIGHBORS\_REQUEST (BRDCST) & 0x08 & ORIGINUID \\
    FIND\_NEIGHBORS\_RESPONSE & 0x09 & SRCUID, ORIGINUID, SRCNAME \\
    ACK FIND\_NEIGHBORS\_RESPONSE & FIND\_NEIGHBORS\_RESPONSE & ACK & SRCUID, DESTUID \\
    FRIEND\_REQUEST & 0x0A & SRCUID, DESTUID, SRCNAME \\
    ACK FRIEND\_REQUEST & FRIEND\_REQUEST & ACK & SRCUID, DESTUID, SRCNAME \\
    FRIEND\_RESPONSE & 0x0B & ACK|ACCEPT & SRCUID, DESTUID \\
    ACK FRIEND\_RESPONSE & FRIEND\_RESPONSE & ACK & SRCUID, DESTUID \\
    UNFRIEND\_REQUEST & 0x0C & SRCUID, DESTUID \\
    ACK UNFRIEND\_REQUEST & UNFRIEND\_REQUEST & ACK & SRCUID, DESTUID
  \end{tabular} \linebreak
  \caption {Base-to-Base Packet Structure}
%\end{table*}
\label{Packet Table}
\end{sidewaystable*}

Table \ref{Packet Table} lists the entire packet structure for the base-to-base communications.  It is set up to take under 100 Bytes to transmit the entire payload for each packet type, which is the maximum payload of a ZigBee transmission using our transceiver. Any size greater will cause packet fragmentation, in which we would have to reassemble the packets at the destination. An in-depth description of all of the packets can be seen in Appendix \ref{Packet Descriptions}.

The packets which allow users to connect to one another include the FIND\_NEIGHBORS\_REQUEST, which polls the area for contacts.  This will allow the user to locate their individual friends who are nearby as well as individually connect to people who are not within range, providing they are on the same network.

The FRIEND/UNFRIEND packet allows friends to connect to a base station and provides for help requests to be sent.

The main packet driving the system's purpose is the HELP\_REQUEST packet.  If two devices are paired, they can send these packets to one another.  When a help request is sent, it is immediately acknowledged, flags are set within the paired system, and a call to action takes the form of various peripherals with flashing lights and generated noise to signal the friend has requested assistance.  The only way to turn it off is to wait for it to timeout or respond with an accept/reject acknowledgement.  If it does timeout, additional friends on the list are contacted until the list is exhausted.

The HELP\_FROM\_ANYONE Request, which in conjunction with the FIND\_HOPS packet types, allows the base station to request help based on proximity to the station.  This is only used if no friends can answer the original alert request.

\subsubsection{Basecomm}
The base communication file is the packet conversion tool which converts data from the network into packets, as well as packets into data which can be sent on the network. Each converting function prepares all of the required fields based on the packet type, and a helper function copies all of the necessary fields into the application packet. This was originally developed during the Linux testing phase, where we used a different networking protocol, TCP/IP, to send these packets over the network. This was somewhat successful in the Linux Version, as the Network Adaptation Layer to simulate a Zigbee-like network was written only worked on a specific Ubuntu virtual machine. This however proved the concept of the Network Adaptation Layer, which we used in the final build of the system, using the actual hardware rather than a Linux simulation.

\begin{figure}[ht] 	% There are several different modifiers that can be used in [].
\centering
\includegraphics[width=0.4\textwidth]{MessageQueueFlowchart.png}
\caption{ \space Flow of packets to be processed and sent through the message queue system.}
\label{Message Queue Flowchart}
\end{figure}

\subsubsection{Message Queue} 
The goal of the message queue system was so messages sent throughout the network to different stations would be validated before being processed. This means that anyone that connects to the network, including malicious or disruptive users, would have their data validated, and minimize the effects of said malfunctioning devices or malicious users. As shown in Figure \ref{Message Queue Flowchart}, it can be seen that the receive queue of the Message Queue system first inspects the packet for any network issues. If the packet was destined for the device, and the device is expecting the packet, it leads into the Packet Processor. The packet processor has the ability to process as well as send out packets, as well as take input from the user or timer events to refresh old data and test connections between other stations. These items are then sent to the send queue, which will then send it out to the network to other base stations. Though not all of the functions in messagequeue we envisioned ended up being fully implemented, it was a good way to access the packet flow of the system, and to make it less likely to have unnecessary complications later down the road. The main function that was not implemented in the message queue system was the reloading of the system packet responses upon a crash or reboot, which I explain later in the storage section.
\subsubsection{Packet} 
The packet file is used to handle all of the incoming packets, as well as having functions to send packets, which may be called by the user interface, packets, or as a timer event. All of the messaging state processing is handled in this file. The handlers are registered in a function pointer array, and then processed into application packets via a packet converter, which parses the raw packet data from the network layer into an application packet.\newline
Each handler is given the packet, which contains the packet information, as well as an identification number for the network information from which the packet was received from. Some of this data is the time to live on the packet, the source address and destination address. The time to live can be used to determine the number of hops, or physical network devices it took for the packet to arrive at the station, which is what we use it for.
\subsubsection{Neighborlist} 
The purpose of the neighbor list module is to allow for the device to populate a list of nearby base stations, also known as neighbors, which would allow for the user to become friends with those different stations, and thus their users. This list is displayed on the graphical user interface after the neighbors are polled using the find neighbors packet types. To connect to their neighbors, they would go to the add friend menu, which would populate the list with all stations on the network. The user can then send friend requests to these users. This system can also be used to probe the network for the purposes of minimizing large overly large numbers of broadcasts, to keep any broadcasts as physically local to the station, to improve network stability as well as reduce excessive alarms in the help from anyone request, which has the potential to reach a large amount of people.
\subsubsection{Friendlist} 
The friend list is used to create a contact list so that the user can send friend requests to other stations. This is a list of friends that will be contacted upon a help request, and it will send it to them one at a time, in order of the priority that the user assigns them to. The uses of the friends list can be expandable in the future, such as sending private messages to your friends. For now though it is limited to sending help requests. This file also contains code which to creates the local user list. This list is for people that are connected directly to this base station. For instance, if a couple both had their own button, they could have it connected to the same base station, and still have their own name being displayed in events such as a help request. The local user is an extension of the friend struct, both of which can be seen in Appendix \ref{Friend Data Structure} and Appendix  \ref{LocalUser Data Structure}. The other data that is stored besides their own “Friend” data is related to their phone number and home address, which is sent over the network on a help request. The reasoning behind sending these addresses only when they request for help is a privacy concern. By only sending it when it is absolutely necessary, they can avoid the guest of other neighbors snooping through their contacts list to get mailing addresses of their friends, or through network sniffing. This limits this interaction to only when the person actually needs help, where it is more important that they can get the help they need when they need it.
\subsubsection{Storage} 
In the event of a device reset, whether it be from events such as software crashes and power fluctuations, it would be possible to restore the friend status and important pending packet events, such as help requests and friend requests. This system would store response flags in the friendlist, which would determine what current packets are being expected, such as a help request that has been sent to them. The framework for this system is in the code, but it was never implemented, as we didn’t have a way to store the data, as we were getting low on available hardware pins to drive the other peripherals. This means that in its current form, it will not save user preferences or the fail-safe data, which will need to be rectified in a future revision, with the addition of non-volatile memory.
\subsubsection{Alarm}
The purpose of the alarm module is to physically alert the user using the sound and lighting peripherals. I setup the software for the alarm to be able to support multiple alarm priorities, which will allow for higher priority messages such as help requests to override lower priority items. For instance, if the user was using the lighting component as a desk lamp, it would switch over to the higher priority flashing mode upon receiving a help request, which you can see in Figure \ref{Alarm Priorities}
\begin{figure}[ht] 	% There are several different modifiers that can be used in [].
\centering
\includegraphics[width=0.4\textwidth]{AlarmPriority.png}
\caption{ \space Potential lighting states based on priority.}
\label{Alarm Priorities}
\end{figure}

\subsection{Application Layer (Linux Base-to-Base Emulation)}
Our intentions in creating the Linux base-to-base emulation of our system was due to not having access to the hardware in a timely fashion.  The application layer was divided into different files for simplicity and organizational purposes.  It emulates the menu items displayed on the screen (as would be seen by a user) and it emulates a network in which a device is connected.  This Linux version of the system simulates sending a help packet to the contacts listed in a user’s friends list.  A packet will be sent to the first contact on the list, it will timeout if the help request is not acknowledged as received, at which time it will be sent to the next contact on the list.

\subsection{Hardware}

\subsubsection{Button}

\textbf {Microcontroller}

TODO: Missing figure

The microcontroller design layout was based on recommended considerations from relevant application notes and datasheets.  The BlueNRG-1 uses two external oscillators with ambiguous capacitor values in this first revision diagram to be clarified after further testing.  The microcontroller layout will be on a separate breakout board with an attached antenna [as well as all other components displayed above] to allow for faster, more consistent prototyping.  However, our early revisions connect to antennas via an edge-mounted SMA connector.  This layout must allow for flexibility of GPIO assignments, so the majority of the pins are mapped to corresponding headers on the attached power/peripheral board even if currently out of use.

Programming will be completed over 5 JTAG-specified pins.  During layout, great considerations have been taken to minimize the distances between high-speed components, particularly the bypass capacitors.

\textbf {Power:}

TODO: Missing figure

The basic power system for the button needs to be stable and immune to noisy environments.  Our next revision will aim to add complexity and at least include a forward-biased diode at the cathode of the battery.

\textbf {Peripherals:}

TODO: Missing figure

Currently, only the button and LED indicator will be added to the button.  After functionality is proven, we will reconsider adding an accelerometer.\\

\textbf {Connectors:}

Connectors have been added to allow for connection to the microcontroller’s separate breakout board.  Other connectors include JTAG links and test points (not shown here).

\textbf {Parts List}

\begin{itemize}
  \item ST BlueNRG-1 microcontroller
  \item 1 x 1000mAh, 3V Coin cell battery
  \item 1 x Push-button
  \item 300 x Female headers
  \item 300 x Male headers
  \item 500 x Assorted surface mount resistors and capacitors
\end{itemize}

\subsubsection{Base Station}

TODO: Missing Figure

TODO: Missing Figure

Parts List
\begin{itemize}
  \item 1 x Atmel SAMB11 microcontroller
  \item 1 x XBee-Pro 915 MHz module
  \item 1 x 26 MHz crystal
  \item 1 x 32.768 kHz crystal
  \item male/female pin headers
  \item Assorted surface-mount resistors and capacitors
\end{itemize}

\textbf {Layout}

The base station unit is the other crucial component to this decentralized ad hoc neighbor system. Each household contains a single base station unit that pairs with its own unique button. When the button signals a help request, the corresponding base station receives the signal and broadcasts the help request to nearby friendly base stations in search of a response. The help signal is recursively rebroadcasted by the receiving base stations to any immediate base stations within their broadcast range, allowing for a theoretical unlimited search range. This work-around allows the system to get past the limited radiation power set under the FCC Part 15 regulations for unlicensed low-power transmitters.

The base station consists of a bottom-up stack of PCBs. The bottom board contains the power supply, battery backup, as well as the amplifier and switching circuit for the speaker and light. The microcontroller sits on the board just above with both the 2.4GHz and 900MHz antennas. Female pin headers are extended from this board to allow placement of 16x4 character LCD display PCB just above.

\begin{figure}[ht] 	% There are several different modifiers that can be used in [].
\centering
\includegraphics[width=0.4\textwidth]{base-schematic-uc.png}
\caption{ \space Base Station Schematic: Microcontroller}
\label{base-sch-uc}
\end{figure}

The Atmel SAMB11 system-on-chip is the base station's microcontroller. It features a quad-flat-no-lead (QFN) packaging with 48 pins and a pitch of 0.4mm distance between pins on a single side. The microcontroller provides an on-chip Bluetooth Low Energy module that allows easy access to the 2.4GHz radio band. The microcontroller operates with two external crystal oscillators: a 32.768kHz real time clock and a 26MHz normal operation clock. The SAMB11 operates on a +3.3V voltage rail for its digital core, but it also contains an internal DC-DC converter to +1.2V for radio operations.

The microcontroller has a total of 27 GPIO pins, where 4 pins have mixed-signal capabilites and the rest are standard digital. The SAMB11 has an onboard analog-to-digital converter (ADC) accessible by these mixed-signal GPIOS, but our system does not have use of this. Instead, these GPIOs are used as typical digital pins. All onbaord GPIOs have internal resistors, whose pull orientation can be programmed.

The onboard Bluetooth module is accessed via the RFIO pin, which handles both the receiving and transmitting radio ends. Our system provides a patch antenna tuned to 2.4GHz operation with a matching network of transmission lines. Unlike the button, the base station is not limited to be as small as possible. Therefore, instead of using a more compact, electrically-short chip antenna that has limited performance, the system takes advantage of the extra real estate with a higher-performance patch antenna.

The controller is programmed via serial wire debug (SWD), which requires 5 signals for operation: voltage reference, IO, clock, reset, and ground. The IO and clock is connected directly to the corresponding pins on the microcontroller, while the reference voltage and ground is connected to the digital core voltage of +3.3V and the microcontroller ground, respectively. The SWD reset pin connects to the microcontroller's CHIP\_EN, which is required to stay high for normal operations. The debugger momentarily pulls CHIP\_EN low to restart the system. While the GPIOs have internal pull up/down resistors, an external pull up resistor must be placed by CHIP\_EN to ensure an operational chip in the absence of the SWD debugger.

\begin{figure}[ht] 	% There are several different modifiers that can be used in [].
\centering
\includegraphics[width=0.4\textwidth]{base-schematic-xbee.png}
\caption{ \space Base Station Schematic: XBee Transceiver}
\label{base-sch-xbee}
\end{figure}

The Digi XBee radio module is the choice of transceiver to communicate on the 900MHz band. An external antenna can be connected through its onboard SMA receptacle. The XBee communicates with the microcontroller via UART with 4 pins: receiving data (RX), transmitting data (TX), clear-to-send (CTS), and request-to-send (RTS).

\begin{figure}[ht] 	% There are several different modifiers that can be used in [].
\centering
\includegraphics[width=0.4\textwidth]{base-schematic-lcd.png}
\caption{ \space Base Station Schematic: Display}
\label{base-sch-lcd}
\end{figure}

A 16x4 character LCD display is chosen for user interface to the base station system. It provides a larger space for viewable data than its 16x2 character counterpart, while remaining relatively inexpensive in comparison to other larger display screens. The LCD is the only device on the board that operates on a +5V rail as opposed the microcontroller's +3.3V rail. The register select (RS), read/write, and enable pins control the data flow set by the 8 data pins of the LCD. To conserve on limited GPIO resource, the LCD was set to half-byte operation, reducing the necessary data pins down to 4. The ability to view the LCD depends on its character contrast and the backlight brightness. A trimpot resistor allows for contrast tuning based on the user's viewing angle, while the LCD backlight is controlled via a PWM signal from the the microcontroller to a power MOSFET.

\begin{figure}[ht] 	% There are several different modifiers that can be used in [].
\centering
\includegraphics[width=0.4\textwidth]{base-layout-uc.PNG}
\caption{ \space Base Station Layout: Microcontroller}
\label{base-lay-uc}
\end{figure}

Layout of the microcontroller on the base station demands careful consideration due to the tiny nature of the chosen components, which are sized in the sub-centimeter range. The biggest concern is the noisy nature of any electromagnetic interference (EMI). This interference can be found everywhere, but on a PCB, the main sources can be found in crystal oscillators, power supplies, and antennas. The high frequency components found within these elements is the origin to this noisy behavior.

As such, crystal placement is of utmost importance. While the crystal can inflict noise onto nearby signal traces, the oscillator itself is also very prone to noise. The microcontroller requires a clean oscillating signal to operate properly with its clocking signal. Placing the crystals as close as possible to the microcontroller allows for short connecting traces and reduces the possibility of defect in the oscillating signal. In addition, the 26MHz crystal has a ground pad, which must have a low impedance short return path to the microcontroller. By low impedance, we mean a minimally-disrupted wide path that allows current to flow relatively freely. This low impedance short path reduces the ground loop in which current flows, lowering the noisy effect on nearby traces. To further reduce any noise coupling, signal traces should avoid as best as possible the immediate vicinity of the crystals.

The analog signal through RFIO is the Bluetooth data signal that is handled by our onboard patch antenna. The feedline trace for the antenna is relatively isolated from any components in the same reason traces avoid areas near the crystals. Nearby components will always carry some sort of EMI from the stray capacitance and inductance. This semi-isolation tactic reduces the chance of noise but also creates a clean analog ground return to the microcontroller.

\begin{figure}[ht] 	% There are several different modifiers that can be used in [].
\centering
\includegraphics[width=0.4\textwidth]{base-layout-full.PNG}
\caption{ \space Base Station Layout: Radio and Display}
\label{base-lay-full}
\end{figure}

Because the base station operates on two different frequency bands, the 2.4GHz and the 900MHz, the layout must include sections for the antennas. Placement of these elements should be as far away as possible to avoid any unnecessary couplilng, as antenna behavior is heavily affected by its surroundings. The patch antenna is used for broadcasts in the 2.4GHz domain, and it has a shorted transmission line stub to match the input impedance of the microcontroller's RF end with the antenna's characteristic impedance. On the other side, the Digi XBee module is used for broadcasts in the 900MHz range. The XBee takes digital UART signals from the microcontroller before passing through its own onboard gain and modulation stage for the mesh network signals. As a result, we do not have to strictly follow the same analog isolation rules as the 2.4GHz signal. The XBee has an onbard SMA receptacle that allows us to place an external antenna, which can be routed to a better reception position and angle. Initial prototypes used an off-the-shelf half-wave dipole connected via a coaxial cable. Future models may replace this component with a cheaper, more sustainable antenna.

The 16x4 LCD display is connected via 16 pin headers. We place the same 16 pin headers in female form so that we may reduce the required land space for placement by instead stacking the LCD right above the XBee module. The extra female headers provide the air space needed to avoid any component collision.

\begin{figure}[ht] 	% There are several different modifiers that can be used in [].
\centering
\includegraphics[width=0.4\textwidth]{base-layout-interface.PNG}
\caption{ \space Base Station Layout: Interface}
\label{base-lay-btn}
\end{figure}



The base station layout requires careful consideration due to the number of components packed into a smaller sized board.  The first board revision for the base station adopts a mother-daughter format as an attempt to reuse many of the tiny expensive components, primarily from the SAMB11 microcontroller.  Analog signal and power components must be placed close to the microcontroller on the daughterboard.  Plated drill holes placed around this section will allow for the daughterboard to be shared and reused.  Future revisions will only include the motherboard, with circuitry outside the microcontroller and its closest components, whereas the older revision using the daughterboard can be mounted directly on the plated drill holes through pin headers.

Component placement is obviously important to reduce the amount of unnecessary electromagnetic interference.  The first point of interest are the two RF signals, where a separate antenna is required for each signal due to their different frequencies.  Since an antenna’s behavior changes according to its environment, it’s important to place these two antennas as far away from each other as possible.  For the SAMB11 2.4 GHz analog signal, the traces leading to the impedance-matching circuit and the antenna should be as short as possible to reduce the effects of stray inductances and capacitances.  The 915 MHz signal is generated by the ZigBee Pro, where data flows through UART pins.  The signals between the SAMB11 and ZigBee stay in the digital domain before being modulated by the ZigBee;  therefore, the transceiver module’s proximity to the microcontroller is of no immediate concern.

Another significant source of noise comes from the crystal oscillator.  The SAMB11 can generate two of its clocks using external crystals: a 26 MHz control clock and a 32.768 kHz real-time clock.  The oscillating currents flowing through the crystal classify it as a heavy noise source.  At the same time, crystals should provide spectrally pure clock frequency for the microcontroller to function optimally and are equally susceptible to external noise.  Important signal traces should avoid the crystal’s vicinity on all copper layers to avoid data contamination from the oscillator.  Additionally, any ground pads on the crystal should have a low impedance ground return path to the microcontroller’s ground pin to ensure as small of a current loop as possible.  This helps reduce the amount of noise which can be seen by other components.

\subsubsection {Power Supply \& Peripherals}

\begin{figure}[ht] 	% There are several different modifiers that can be used in [].
\centering
\includegraphics[width=0.4\textwidth]{BlockDiagram.png}
\caption{ \space Power Supply \& Peripherals Block Diagram}
\label{Psupply}
\end{figure}

The power supply for the base station requires three different output voltages to power the main components. The three output voltages are 12V, 5V, and 3.3V as we can see from the base station power budget. The base station pulls a maximum current of 394.6mA, and a maximum power of 1.82W. The high-level block diagram below illustrates the whole design of the power supply and peripherals. First of all, the base station is powered by a regular 120VAC power outlet that can be found in any home with electricity. A 12V 12W AC/DC external wall mount adapter from Qualtek is used to convert 120VAC to 12VDC. The 12V output from the wall mount adapter is fed to an audio amplifier circuit using a LM368 IC. The output from the wall mount adapter is stepped down to 5V using a linear voltage regulator. The 5V output from the linear regulator is fed to the LCD display as shown in the block diagram. Additionally, the 12V output is stepped down to 3.3V using a Buck Converter to feed the SAMB11 microcontroller, the X-bee pro, and the keypad. 

The power supply also has a 120VAC output that is controlled using a triac. The purpose of the triac is to let us change the output from 120VAC to 0V at any given time. The backup power consists of a 4.8V 700mAh NiMH rechargeable battery. A boost converter is used to step up the 4.8V from the battery to 12V. This allow us to feed 12V to the system like the wall mount adapter so that we can repeat the whole process to power the components in backup battery mode.

\textbf {12-Volt Output: Audio Amplifier }

\begin{figure}[ht]	% There are several different modifiers that can be used in [].
\centering
\includegraphics[width=0.4\textwidth]{Audio.png}
\caption{ 12V Audio Amplifier}
\label{Paudio}
\end{figure}

The 12V output from the wall mount adapter is fed to a LM386 audio amplifier as shown in the circuit diagram above. This means that the output audio signal can swing from 0V to 12V. A 12-Volt Zener diode (D8) is used as a voltage reference to make sure the input voltage rail does not exceed 12V, since it can damage the LM386 IC. However, a 1k$\Omega$ (R11) resistor was added to act as a load in case that the rail exceeds 12V to limit the current looping through the Zener diode. The input signal for the audio amplifier is regulated with a potentiometer using a voltage divider. In this case the gain of the amplifier was set to 200, which is the maximum gain possible. The gain can be reduced by adding a resistor in series with the 10uF capacitor (C5) between pin 1 and 8. Lastly, the output signal goes through a DC coupling capacitor to get rid of any DC components. 


\textbf {5-Volt Output: LCD Display}

\begin{figure}[ht]	% There are several different modifiers that can be used in [].
\centering
\includegraphics[width=0.4\textwidth]{Linear.png}
\caption{ 5V Output for LCD Display }
\label{Pdisplay}
\end{figure}

The LCD display requires a 5-volt input voltage and the data sheet suggests a 5-volt steady input signal to avoid any damages.  An adjustable LM317 linear regulator was used to step the voltage down to 5V.  The output of the linear regulator is given by Vout~= 1.25(1+R3/R4)V, the datasheet suggests keeping R4 at 240$\Omega$ but vary R3 to obtain the desired output voltage.  In this case, R3 is 750$\Omega$, which sets the output voltage to 5.15V.  The capacitor C15 is recommended to prevent any amplification of the output voltage ripple.  The Schottky diode D5 provides a low-impedance path for C15 to discharge so it won’t discharge at the output of the LM317.  A 5-volt Zener diode was included at the output of the LM317 to establish a voltage reference so it would not permit the output to exceed 5V in case of a malfunction.


\textbf {3.3-Volt Output: Microcontroller SAMB11, Keypad, and XBee-Pro}

\begin{figure}[ht]	% There are several different modifiers that can be used in [].
\centering
\includegraphics[width=0.4\textwidth]{Buck.png}
\caption{ Buck Converter 3.3V Output }
\label{PConverter}
\end{figure}

The microcontroller and the XBee-Pro are the most expensive and sensitive devices in the base station. For this reason, a 3.3-volt fixed buck converter LM2574 IC was used, instead of a standard adjustable buck converter.  The values of L1 and CP were chosen carefully so that the switching power supply is in continuous-conduction-mode.  A 3.3-volt Zener diode was added to set up a voltage reference at the output of the buck converter to double-protect the microcontroller and XBee module.  A 500mA fuse was included in case of any short circuits in either the microcontroller or the XBee.  A switch was then included before the input of the buck converter to control the three output voltages, which is not clear from the circuit diagram shown above.

\textbf {120VAC Output: External Light controlled with a Triac}

\begin{figure}[ht]	% There are several different modifiers that can be used in [].
\centering
\includegraphics[width=0.4\textwidth]{Triac.png}
\caption{ 120VAC Output Controlled with a Triac}
\label{PTriac}
\end{figure}

The circuit in Fig. \ref{PTriac} allows the 120VAC output to change to 0V by sending a signal from the SAMB11 microcontroller.  The output is 120VAC when the microcontroller sends a high signal and 0V when the signal is low.  The triac blocks AC electricity when its gate senses no current.  To trigger the triac, the gate must sense a current of at least 25mA.  Unfortunately, the microcontroller could not output 25mA to trigger the gate.  To solve this, a MOC3043M optoisolator was included in the design, which can be activated with a minimum current of 5mA.  The optoisolator also completely isolates any AC currents from the microcontroller, which is essential, since AC currents and voltages can damage the microcontroller and the XBee-Pro.  The purpose of this circuit is to allow the user to plug any AC lamp into the base station to use as an alert, flashing in case of an emergency.  Note that this circuit will not operate in backup power mode.


\textbf {Backup Battery}

\begin{figure}[ht]	% There are several different modifiers that can be used in [].
\centering
\includegraphics[width=0.4\textwidth]{Boost.png}
\caption{Backup Battery with Boost Converter}
\label{Pboost}
\end{figure}

The options for backup battery were limited by size, cost, capacity, and safety.  A 4.8V 700mAh NiMH battery met the criteria, however the output voltage of the battery is not compatible with the 12-volt main power rail.  A boost converter was used to step the voltage up from 4.8V to 12V.  This allowed the reuse of the wall-mount adapter step-down voltage process, which reduced complexity in the system.  The boost converter used is a MC33063A adjustable IC, the output voltage is given by Vout=1.25(1+R10/R9).  R9 was set to 1k$\Omega$ and R10 to 10k$\Omega$.  Thus, the output of the boost converter is 13.75V when the battery is fully charged.  However, the voltage of the battery drops when discharging so it is convenient that the output of the boost converter is greater than 12V much of the time during discharging.  The Schottky diodes D2, D12, and D4 are included to allow current flow in one direction to avoid any unwanted currents going to the boost converter or the battery.


\textbf {Future Revisions of Power Supply}

It was discovered with the final revision that output voltages were not the required values.  For example, the output for the display was 3.7V instead of 5V, and the output of the boost converter is not 12V.  To correct this, different resistances must be tested experimentally until the desired output is achieved.  One of the primary concerns is a charging system design for the backup battery.  This charging system must be well designed to optimize the battery life and for the safety of the user, since batteries can be unstable at hot temperatures.  Finally, a noise filter must be designed for the audio amplifier because the output audio signal contains too much noise so static is audible without an audio input signal.


\subsubsection {Antennas}

\textbf {Button}

The antenna for the button needed to operate in the chosen frequency band of 2.4 GHz and it was required to be compact so it can be contained within the button, which is to be worn as either a for a necklace or a bracelet pendant.  This size constraint required careful consideration to produce the optimum performance, which was a primary motivating factor to use this high-frequency band.  The minimum transmission range for the button is the approximate length of our customers’ homes at 50 feet and it must communicate with the base station in the home environment.  This means the button antenna needs to communicate through walls and despite other types of interference such as microwaves.

Nine different PCB antenna layouts were initially designed based on proven engineering models, utilizing appnotes and datasheets as guides.  Subsequent revisions were created to properly tune the antennas.  Antenna testing is a time consuming and meticulous process.  For lab testing, a high-frequency network analyzer was used to measure impedances with the baseline impedances measured first.  The SimSmith Smith chart simulation software was then used as a guide for designing impedance matching networks for optimum power transfer between the antennas and button circuit at the ideal impedance of 50$\Omega$ with no reactants.  A radio-frequency link was created in the lab and a high-frequency spectrum analyzer was used to measure the far field performance of each antenna.  The far field is where the electromagnetic waves are free to radiate without the influence of the moving charges which produced them, once the waves have traveled far enough away from those charges.  In contrast, the near field is where these radiating waves are close to the charges and the current which created them.

Proper testing requires the environment to remain identical for each test.  Each antenna also needed to be tested in a variety of positions relative to the transmitted ground plane reference signal.  This means everything in the lab would ideally be in the same exact location for each test.  The far field lab testing measured the link loss of each antenna to determine which designs had the best performances relative to one another.   These antennas not only needed to radiate up to 50 feet but they needed to work properly in any position and behind obstacles.  The initial testing used a radio-frequency link distance of 2 meters.  Then based on these results, it was determined that the small MIFA and chip corner-mounted designs performed best relative to the other designs.  They had better baseline impedances and less link loss.  These two antennas were then tuned with the appropriate impedance matching networks in the next revisions.  Finally, the antenna chosen for the button was the small MIFA since it performed approximately as well as the chip antenna and was smaller, making it a better fit for use in the button device.


\begin{table*}[H]
  \centering
  \begin{tabular}{>{\bfseries}l|l l l l}
  \hline
    Topology & \multicolumn{1}{|c|}{2.39 GHz} & \multicolumn{1}{|c|}{2.43 GHz} & \multicolumn{1}{|c|}{2.46 GHz} & \multicolumn{1}{|c|}{2.50 GHz} \\
    \hline
    MIFA (Slanted Feedline, Small) & \multicolumn{1}{|r|}{42.92 - j4.2, 16.7 pF} & \multicolumn{1}{|r|}{34.1 - j13.8, 5.1 pF} & \multicolumn{1}{|r|}{27.6 - j15.1, 4.4 pF} & \multicolumn{1}{|r|}{22.5 - j16.4, 3.8 pF} \\
    MIFA (Small) & \multicolumn{1}{|r|}{44.2 + j26.9, 1.8 pF} & \multicolumn{1}{|r|}{67.3 + j29.8, 1.9 pF} & \multicolumn{1}{|r|}{88.6 + j11.9, 745 pH} & \multicolumn{1}{|r|}{89.4 - j9.6, 7.7 pF} \\
    MIFA (Slanted Feedline, Large) & \multicolumn{1}{|r|}{39.1 + j8.5, 555 pH} & \multicolumn{1}{|r|}{30.0 + j13.1, 857 pH} & \multicolumn{1}{|r|}{27.5 + j13.7, 896 pH} & \multicolumn{1}{|r|}{22.4 + j13.9, 880 pH} \\
    MIFA & \multicolumn{1}{|r|}{27.9 + j6.1, 411 pH} & \multicolumn{1}{|r|}{20.6 + j8.2, 532 pH} & \multicolumn{1}{|r|}{18.0 + j8.8, 567 pH} & \multicolumn{1}{|r|}{10.5 + j11.1, 692 pH} \\
    IFA (Left Turn) & \multicolumn{1}{|r|}{17.7 - j36.0, 1.84 pF} & \multicolumn{1}{|r|}{18.1 - j36.0, 1.81 pF} & \multicolumn{1}{|r|}{18.2 - j36.2, 1.78 pF} & \multicolumn{1}{|r|}{18.6 - j36.7, 1.73 pF} \\
    IFA (Right Turn) & \multicolumn{1}{|r|}{33.9 + j27.9, 1.86 pH} & \multicolumn{1}{|r|}{32.8 + j37.5, 2.46 pH} & \multicolumn{1}{|r|}{43.5 + j49.3, 3.19 pH} & \multicolumn{1}{|r|}{83.2 + j47.9, 3.01 pH} \\
    Patch & \multicolumn{1}{|r|}{157 + j540, 36.7 nH} & \multicolumn{1}{|r|}{723 + j463, 29.2 nH} & \multicolumn{1}{|r|}{716 - j37.0, 2.1 pF} & \multicolumn{1}{|r|}{420 - j196, 332 fF} \\
    RUFA (Antenova) w/out impedance matching & \multicolumn{1}{|r|}{107.1 - j46.8, 1.2 pF} & \multicolumn{1}{|r|}{76.1 - j60.8, 1.07 pF} & \multicolumn{1}{|r|}{58.4 - j58.6, 1.1 pF} & \multicolumn{1}{|r|}{44.6 - j53.1, 1.2 pF} \\
    Chip corner-mount  w/out impedance matching & \multicolumn{1}{|r|}{32.6 - j43.2, 1.53 pF} & \multicolumn{1}{|r|}{31.2 - j39.6, 1.65 pF} & \multicolumn{1}{|r|}{30.2 - j37.2, 1.73 pF} & \multicolumn{1}{|r|}{28.6 - j34.7, 1.83 pF} \\ \hline
  \end{tabular} \newline
  \caption{Bluetooth 2.4 GHz Antenna Impedance Results}
\end{table*}



\textbf {Base Station}

The base station antenna operates in the 900 MHz frequency band to facilitate communication over a longer range of at least 300 feet, from one user’s base station to another.  This lower frequency band is more suitable for long distance radio links since the attenuation suffered by radio waves through free space decreases along with the operating frequency.  The base station antenna must also be capable of handling atmospheric loss due to gases and be able to navigate through outdoor environments, such as trees and other obstructions.  One of the biggest advantages of this frequency band is that it’s “Near Line of Sight”, meaning better range can be achieved in this band than other ISM bands given the same obstructions.  Since there isn’t as much of a size constraint as with the button antenna, the base station antenna can be larger and external.  At this point we have created two different PCB base station antenna layouts and have a proven whip antenna as a reference, and we will have those soon for testing.


\subsection{Button to Base Application}

The button-to-base station communication is programmed using the General Access Profile (GAP) layer and the Generic Attribute Profile (GATT) layer of Bluetooth Low Energy.  Accessing the API’s listed in bluenrg1\_api.h allows for the utilization of advertising packets over the air according to Bluetooth standards.  The GAP layer defines the basic requirement of a Bluetooth device.  It describes the device's role, its different modes, and its procedures for its mode to enable a connection between Master and Slave devices.  Once the GAP has been initialized with its role as a peripheral, its mode is set to be connectible to any device or in Generally Discoverable mode and its procedure to this mode is to establish a connection to any device that tries to connect to it then it could continue establishing security parameters for pairing. The ADV\_IND flag is used for General Discovery mode. The justification behind the General Discovery mode and its respective procedure is that in case the advertising button is lost or damaged the user could easily reestablish connection with the new advertising button with no technical help. When not advertising the BlueNRG-1 is then listening for a connection response from the initiator, the Master device. Once connected the Slave device sends a slave security request to the SAMB11, which then the Security Manager handles any pairing transactions. Before any connection takes place the scanning device, the SAMB11 must also be configured to be a central device.

The Atmel SAMB11’s GAP layer must also be initialized, which is accomplished using the API's listed in at\_ble\_api.h.  Its role is set as a central device with a General Discovery procedure. The General Discovery procedure allows the central device to scan the surrounding advertising devices. Again, this is so that in case the user loses or damages the advertising button the user could easily reconnect the new button to the central device. The AT\_BLE\_SCAN\_GEN\_DISCOVERY flag is used to scan all advertising Bluetooth devices. To guarantee that he central device only connects to the intended advertising button, the SAMB11 scans the local names of the advertising devices. As it scans if it finds the string "BlueNRG1\_Chat" it has successfully found the BlueNRG-1 advertising. This string can be changed on the BlueNRG-1 to another appropriate name. Once found the connection process is initiated by the SAMB11. If authentication and encryption is used then as the SAMB11 scans the BlueNRG-1's slave security request will come with the advertised payload of the packet sent.

The Security Manager authenticates the connection between the Master and Slave devices by exchanging the 128-bit Long-term-keys (LTK) generated by the AES algorithm. Unfortunately, because the advertising button does not have a display to enter a passkey, the encryption method used is the "Just Works" pairing method for both devices. This means that the connection is exposed to Man-In-The-Middle (MITM) attacks. An argument could be made that not much security is needed since all that would be transmitted between these two devices is "Help Request" packets and battery voltage levels. On the press of a button the BlueNRG-1 sends a connection request to the SAMB11. When the SAMB11 responds an exchange of LTKs are then exchanged to pair both devices together.  When both devices finish exchanging the LTKs, pairing is complete and the device's transition from the GAP layer to the GATT layer. When entering the GATT layer the roles implicitly change from Master\textbackslash Slave to Server\textbackslash Client.

The Server is the device that has updated data to send and the Client is the device that stores and processes this information. In this case the BlueNRG-1 is implicitly the Server and the SAMB11 is the Client. The BlueNRG-1 is the server because it has the voltage level that the base station needs to worry about and it is the device that sends the "Help Request" data the base station needs to process. Before the BlueNRG-1 enters the GATT layer it already has configured a table shown in Table \ref{BT Access}. 

\begin{table}[H]
\centering
\caption{Bluetooth Profile Table}
  \begin{tabular}{|l|c|c|c|c|}
      \hline
      Description         &Handle& UUID              & Value\\
      \hline
      Chat Service 	      & 0x0C & 0xD973F2E0...66   & \\
      \hline
      TX Characteristic   & 0x0D & 0xD973F2E1...66   & \\
      \hline
      Notif Handle	      & 0x0E & 				     & uint8*\\
      \hline
      CCCD                & 0x0F & 			         & \{0, 1\}\\
      \hline
      RX Characteristic   & 0x10 & 0xD973F2E2...66   & \\
      \hline
      Write Handle		  & 0x11 &					 & uint8*\\
      \hline
      Battery 		      &		 &					 & \\
      Characteristic	  & 0x12 & 0xD973F2E3...66   & \\
      \hline
      Read Handle		  & 0x13 & 					 & uint8*\\
      \hline
  \end{tabular}
  \label{BT Access}
\end{table}
A Bluetooth Profile may consist of the many services, characteristics, and attributes. In the profile created for the button to base communication only one service is created that contains three characteristics.  The handles give order to the table in organization of service, characteristic, and attribute order. Each characteristic has a description of itself and it identifies itself through the UUID. The UUIDs in the button to base profile are all custom characteristics therefore, they all must be 128-bits.  Generic characteristics are 16-bits. Each characteristic must also have a property that gives both devices the ability to send and receive data that modifies their respective attribute values.\\

Now in the GATT layer, to be able to chat back and forth between the BlueNRG-1 and SAMB11 the "TX Characteristic" and "RX Characteristic" were made. These are just names of course to what actually allows the transactions of data to take place. The BlueNRG-1 exposes the characteristics shown in Table 8. The SAMB11 then goes through the discovery process of these characteristics. Once discovered, both devices have the same characteristics with the same handles stored. The SAMB11 will discover that the TX characteristic contains the Client Characteristic Configuration Descriptor (CCCD) that comes when the notification property is initiated by the Server. The CCCD attribute value will only take a 0 or 1. The Client will write a 1 to the Server to enable Notifications from the Server. Writing a value of 0 disables Notifications from the Server. When the Client enables Notifications the Server will now be able to send data to the Client through its attribute handle without the need for acknowledgement from the Client. The Client will discover the RX Characteristic, which comes with the Write Property, the handle, and the attribute value. Once discovered the Client would then be able to write to this handle and the Server would then receive the data sent. Lastly, the Client discovers the "Battery Characteristic" along with its Read Property, its handle, and attribute value. The Client could now on command inquire to the Server for some information designated by the Server. In this projects case it would be the battery voltage level being inquired about. Fig. \ref{ble} will help with the visualization of the sequence of events between the Master\textbackslash Slave and the Server\textbackslash Client.
\begin{figure}[H] 	% There are several different modifiers that can be used in [].
\centering
\includegraphics[width=0.4\textwidth]{Bluetooth.png}
\caption{ \space Bluetooth LE sequence of events}
\label{ble}
\end{figure}

Another concern was the power consumption of both the BlueNRG-1 and the SAMB11. To alleviate the radio from being constantly ON. The BlueNRG-1 would be put into sleep mode and wake up every 10 seconds so as not to go into Deep Sleep. In this application the BlueNRG-1 would consume less than 1 $\mu$A and while connected it would consume almost 5 mA. To alleviate the power consumption of the SAMB-11 scanning it would be best reduce the scanning window. So the SAMB-11 would be scanning half the time ON and the other time OFF. It was found that in this application it would consume about 3 mA.
\subsection{Packaging}
Packaging was done using a 3-dimensional modeling software. There were various software that could have been used, but there were tradeoffs between the amount of time to learn one and have a finished packaged product by the end. Various testing and research done on which to use and to name a few there was Solidworks, FreeCAD, and Fusion 360. The most intuitive one to learn in a quick amount of time was Fusion 360 to design the encasements for the base station and the button device. 

Before actually obtaining the software rough sketches were made for the base station and button device. An important thing to keep in mind was to have the button device aesthetically appealing, compact, and jewelry-like. Taking input input of others narrowed down our button device to have two designs in which the customer can choose from and in different colors.The designs for the base station had more freedom to them. Various versions were printed of both the base station and button device as it was difficult to get them right.

% An example of a floating figure using the graphicx package.
% Note that \label must occur AFTER (or within) \caption.
% For figures, \caption should occur after the \includegraphics.
% Note that IEEEtran v1.7 and later has special internal code that
% is designed to preserve the operation of \label within \caption
% even when the captionsoff option is in effect. However, because
% of issues like this, it may be the safest practice to put all your
% \label just after \caption rather than within \caption{}.
%
% Reminder: the "draftcls" or "draftclsnofoot", not "draft", class
% option should be used if it is desired that the figures are to be
% displayed while in draft mode.
%
%\begin{figure}[!t]
%\centering
%\includegraphics[width=2.5in]{myfigure}
% where an .eps filename suffix will be assumed under latex,
% and a .pdf suffix will be assumed for pdflatex; or what has been declared
% via \DeclareGraphicsExtensions.
%\caption{Simulation results for the network.}
%\label{fig_sim}
%\end{figure}

% Note that the IEEE typically puts floats only at the top, even when this
% results in a large percentage of a column being occupied by floats.


% An example of a double column floating figure using two subfigures.
% (The subfig.sty package must be loaded for this to work.)
% The subfigure \label commands are set within each subfloat command,
% and the \label for the overall figure must come after \caption.
% \hfil is used as a separator to get equal spacing.
% Watch out that the combined width of all the subfigures on a
% line do not exceed the text width or a line break will occur.
%
%\begin{figure*}[!t]
%\centering
%\subfloat[Case I]{\includegraphics[width=2.5in]{box}%
%\label{fig_first_case}}
%\hfil
%\subfloat[Case II]{\includegraphics[width=2.5in]{box}%
%\label{fig_second_case}}
%\caption{Simulation results for the network.}
%\label{fig_sim}
%\end{figure*}
%
% Note that often IEEE papers with subfigures do not employ subfigure
% captions (using the optional argument to \subfloat[]), but instead will
% reference/describe all of them (a), (b), etc., within the main caption.
% Be aware that for subfig.sty to generate the (a), (b), etc., subfigure
% labels, the optional argument to \subfloat must be present. If a
% subcaption is not desired, just leave its contents blank,
% e.g., \subfloat[].


% An example of a floating table. Note that, for IEEE style tables, the
% \caption command should come BEFORE the table and, given that table
% captions serve much like titles, are usually capitalized except for words
% such as a, an, and, as, at, but, by, for, in, nor, of, on, or, the, to
% and up, which are usually not capitalized unless they are the first or
% last word of the caption. Table text will default to \footnotesize as
% the IEEE normally uses this smaller font for tables.
% The \label must come after \caption as always.
%
%\begin{table}[!t]
%% increase table row spacing, adjust to taste
%\renewcommand{\arraystretch}{1.3}
% if using array.sty, it might be a good idea to tweak the value of
% \extrarowheight as needed to properly center the text within the cells
%\caption{An Example of a Table}
%\label{table_example}
%\centering
%% Some packages, such as MDW tools, offer better commands for making tables
%% than the plain LaTeX2e tabular which is used here.
%\begin{tabular}{|c||c|}
%\hline
%One & Two\\
%\hline
%Three & Four\\
%\hline
%\end{tabular}
%\end{table}


% Note that the IEEE does not put floats in the very first column
% - or typically anywhere on the first page for that matter. Also,
% in-text middle ("here") positioning is typically not used, but it
% is allowed and encouraged for Computer Society conferences (but
% not Computer Society journals). Most IEEE journals/conferences use
% top floats exclusively.
% Note that, LaTeX2e, unlike IEEE journals/conferences, places
% footnotes above bottom floats. This can be corrected via the
% \fnbelowfloat command of the stfloats package.

\section{Conclusion}

Although the design goal of less than a \$100 product cost was not met, substantial progress was achieved with a \$150 product cost. Discussions regarding how to reduce the cost have been focused on replacing the Digi XBee module with a less expensive alternative or using an available off-the-shelf system on a chip (SOC) device.

After the design decisions about the new system are complete, a study at the DeAnza elderly community will occur. This study will test the system and it's design, particularly <TODO!! complete>

%
% if have a single appendix:
%\appendix[Proof of the Zonklar Equations]
% or
%\appendix  % for no appendix heading
% do not use \section anymore after \appendix, only \section*
% is possibly needed
%
% use appendices with more than one appendix
% then use \section to start each appendix
% you must declare a \section before using any
% \subsection or using \label (\appendices by itself
% starts a section numbered zero.)
%
\appendices

\section{Packet Descriptions}
\label{Packet Descriptions}
\begin{LaTeXdescription}
  \item[PING REQUEST]
  The Ping Request message will request the destination to send a return packet to identify that it is still on the network. For a successful request, the source must know the destination UID and transmit it to that node.
  ACK should be set to 0.
  \item[ACK PING REQUEST]
  The ACK Ping Request message will return a packet to the source. It will also return the responders UID and Name.
  ACK flag set to 1.
  \item[HELP REQUEST]
  The Help Request message will be used to request help from a previously friended node. When sending the initial message, the IMM flag may be used if the user wants the friend to request immediate assistance from emergency services.
  After the initial message, the CANCEL flag may be used to cancel any requests that have been made previously. The Home Address and Phone data packets should not be transmitted in this case.
  Set the ACK flag to 0 for all requests.
  \item[ACK HELP REQUEST]
  On the receiving end, the request will be acknowledged. This will tell the requester that the message has been received. The CANCEL and IMM flag should be transmitted the same as was received. The responders UID is also transmitted in the process.
  Set the ACK flag to 1.
  \item[HELP RESPONSE]
  The responder will either set the ACCEPT flag to 1 to accept a request or reject it by setting it to 0. It will also return the responder UID. 
  Set the ACK flag to 0.
  \item[ACK HELP RESPONSE]
  The requester will acknowledge the help response by sending back the packet with the same ACCEPT flag and the ACK flag set to 1.
  \item[HELP FROM ANYONE REQUEST (Broadcast)]
  The Help ALL Request message will be used to request help from any nearby node. When sending the initial message, the IMM flag may be used if the user wants the friend to request immediate assistance from emergency services.
  After the initial message, the CANCEL flag may be used to cancel any requests that have been made previously. The NAME data packet should not be transmitted in this case.
  This message does not need to be ACKnowledged.
  While the TTL is still greater than 0, this same packet is rebroadcast with the TTL decremented. The TTL should also be signalled to the network layer where it will be implemented.
  \item[HELP FROM ANYONE RESPONSE]
  The help all response is to be used to accept the HELP ALL REQUEST broadcast message. The ACCEPT flag will be set based on if the responder accepts or rejects the request. 
  The ACK flag should be set to 0.
  \item[ACK HELP FROM ANYONE RESPONSE]
  The help all response ACK is used to acknowledge the response. If the response was accepted, the home address and phone number will be transmitted. Otherwise, they will not be transmitted. 
  The ACK flag should be set to 1.
  \item[FIND HOPS REQUEST (Broadcast)]
  This message is for polling nearby hops to be checked for. 
  The ACK flag should be set to 0.
  While the TTL is still greater than 0, this same packet is sent with the TTL decremented. The TTL should also be signalled to the network layer where it will be implemented.
  \item[FIND HOPS RESPONSE]
  This message is to respond to the poll of nearby hops.
  The ACK flag should be set to 0.
  The TTL should be sent back for configuration purposes.
  It is expected that the network layer can identify the source of the broadcast from the hops request.
  \item[ACK FIND HOPS RESPONSE]
  This message is to ACK the response to the poll of nearby hops.
  The ACK flag should be set to 1.
  \item[FIND NEIGHBORS REQUEST (Broadcast)]
  This message is for polling nearby neighbors to be listed as potential friends.
  While the TTL is still greater than 0, this same packet is sent with the TTL decremented. The TTL should also be signalled to the network layer where it will be implemented.
  \item[FIND NEIGHBORS RESPONSE]
  This message is to respond to the poll of nearby neighbors to be listed as potential friends. 
  The ACK flag should be set to 0. 
  The TTL should be sent back for configuration purposes.
  It is expected that the network layer can identify the source of the broadcast from the neighbors request.
  \item[ACK FIND NEIGHBORS RESPONSE]
  This message is to ACK the response to the poll of nearby neighbors to be listed as potential friends. 
  The ACK flag should be set to 1.
  \item[FRIEND REQUEST]
  This message is used to request the other user to be a friend.
  The ACK flag should be set to 0.
  \item[ACK FRIEND REQUEST]
  This message is used to ACK the request the other user to be a friend.
  The ACK flag should be set to 1.
  \item[FRIEND RESPONSE]
  This message is the response to the request from the original user.
  The ACK flag should be set to 0.
  \item[ACK FRIEND RESPONSE]
  This message is used to ACK the other user response to be a friend.
  The ACK flag should be set to 1.
  \item[UNFRIEND REQUEST]
  This message is to indicate that they intend to remove them from the friend list.
  The ACK flag is set to 0.
  \item[ACK UNFRIEND REQUEST]
  This message is to respond that they understand that they are removed from the list.
  The ACK flag is set to 1.
\end{LaTeXdescription}

\section{Application Layer Data Structures}
\subsection{Message Queue Code}
\label{Message Queue Code}
\begin{lstlisting}
typedef struct Message {
  opcode opcode; //Expected Packet Response
  uint8_t id; //Internal Friend ID
  bool permanent; //Do not check expiration
  bool broadcast; //Do not check address
  uint32_t expiration; //Expiration time
  char srcAddr[MAXNETADDR]; //Source
  uint16_t srcid;
  uint8_t numUses; //Max. Packets to accept
} Message;
\end{lstlisting}

\subsection{Message Queue Event Data Structure}
\label{Message Queue Event Data Structure}
\begin{lstlisting}
typedef struct {
  char srcAddr[MAXNETADDR];
  uint16_t srcid;
  uint8_t qnum;
} Event;
\end{lstlisting}

\subsection{Friend Data Structure}
\label{Friend Data Structure}
\begin{lstlisting}
typedef struct {
  uint8_t id; //Internal ID in Friendlist
  uint8_t priority; //Call order
  char firstname[MAXFIRSTNAME];
  char lastname[MAXLASTNAME];
  char networkaddr[MAXNETADDR];
  uint16_t port; //Network port
  uint32_t lastresponse; //Timer
  uint16_t responseflag; //16 flags
} Friend;
\end{lstlisting}

\subsection{Packet Data Structure}
\label{Packet Data Structure}
\begin{lstlisting}
typedef struct {
	Friend friend;
	char homeaddr[MAXHOMEADDR];
	char phoneaddr[MAXPHONE];
} LocalUser;
\end{lstlisting}

\subsection{LocalUser Data Structure}
\label{LocalUser Data Structure}
\begin{lstlisting}
typedef struct {
	Friend friend;
	char homeaddr[MAXHOMEADDR];
	char phoneaddr[MAXPHONE];
} LocalUser;
\end{lstlisting}

\section{User Interface Data Structures}

\subsection{Menu Data Structure}
\begin{lstlisting}
typedef struct Menu {
  MenuItem* root;    // Root of menu tree
  MenuItem* current; // Cursor
  // Return value of last callback function
  void* onViewRet;
  void* onClickRet;
  // Buffers for text input
  char** inputBuffer;
  char** outputBuffer;
  // Buffer for network to communicate with menu
  char txtSrc1[10]; 
} Menu;
\end{lstlisting}

\subsection{MenuItem Data Structure}
\label{MenuItem Code}
\begin{lstlisting}
typedef struct MenuItem {
  // Function callbacks
  void* (*onView)();
  void* (*onClick)();
  char value[MENU_MAXLEN];
  bool active;
  // References to other MenuItem's
  MenuItem* parent; // Left
  MenuItem* child;  // Right
  MenuItem* prev;   // Up
  MenuItem* next;   // Down
} MenuItem;
\end{lstlisting}


% use section* for acknowledgment
\section*{Acknowledgment}

The authors would like to thank Professors Ali Adabi, Patrick Mantey, Stephen Petersen, and Anujan Varma for serving as advisors and mentors throughout the development of the project.

The authors would especially like to thank Donald Wiberg, Professor Emeritus at University of California, Santa Cruz, and sponsor of this senior design project.


% Can use something like this to put references on a page
% by themselves when using endfloat and the captionsoff option.
\ifCLASSOPTIONcaptionsoff
  \newpage
\fi



% trigger a \newpage just before the given reference
% number - used to balance the columns on the last page
% adjust value as needed - may need to be readjusted if
% the document is modified later
%\IEEEtriggeratref{8}
% The "triggered" command can be changed if desired:
%\IEEEtriggercmd{\enlargethispage{-5in}}

% references section

% can use a bibliography generated by BibTeX as a .bbl file
% BibTeX documentation can be easily obtained at:
% http://mirror.ctan.org/biblio/bibtex/contrib/doc/
% The IEEEtran BibTeX style support page is at:
% http://www.michaelshell.org/tex/ieeetran/bibtex/
%\bibliographystyle{IEEEtran}
% argument is your BibTeX string definitions and bibliography database(s)
%\bibliography{IEEEabrv,../bib/paper}
%
% <OR> manually copy in the resultant .bbl file
% set second argument of \begin to the number of references
% (used to reserve space for the reference number labels box)
\begin{thebibliography}{1}

  \bibitem{IEEEhowto:kopka}
    H.~Kopka and P.~W. Daly, \emph{A Guide to \LaTeX}, 3rd~ed.\hskip 1em plus
    0.5em minus 0.4em\relax Harlow, England: Addison-Wesley, 1999.

\end{thebibliography}

% biography section
%
% If you have an EPS/PDF photo (graphicx package needed) extra braces are
% needed around the contents of the optional argument to biography to prevent
% the LaTeX parser from getting confused when it sees the complicated
% \includegraphics command within an optional argument. (You could create
% your own custom macro containing the \includegraphics command to make things
% simpler here.)
%\begin{IEEEbiography}[{\includegraphics[width=1in,height=1.25in,clip,keepaspectratio]{mshell}}]{Michael Shell}
% or if you just want to reserve a space for a photo:

\begin{IEEEbiography}{Jamielynne~Batugo}
  Student at UC Santa Cruz.
\end{IEEEbiography}

% if you will not have a photo at all:
\begin{IEEEbiography}{Marco~Carmona}
  Computer Engineering student at UC Santa Cruz.
\end{IEEEbiography}

\begin{IEEEbiography}
[{\includegraphics[width=1in,height=1.25in,clip,keepaspectratio]{profile/clchristensen.JPG}}]{Conrad~Christensen}
  Electrical Engineering student at UC Santa Cruz.
\end{IEEEbiography}

\begin{IEEEbiography}{Jake~Lee}
  Electrical Engineering student at UC Santa Cruz.
\end{IEEEbiography}

\begin{IEEEbiography}
[{\includegraphics[width=1in,height=1.25in,clip,keepaspectratio]{profile/kevinlee.jpg}}]{Kevin~Lee}
  Computer Engineering (Networking) student at UC Santa Cruz. Lead developer tasked with designing and implementing the Network Application Layer and associated protocols.
\end{IEEEbiography}

\begin{IEEEbiography}{Brian~Nichols}
  Computer Engineering student at UC Santa Cruz.
\end{IEEEbiography}

\begin{IEEEbiography}[{\includegraphics[width=1in,height=1.25in,clip,keepaspectratio]{profile/jsoto.jpeg}}]{Jesus~Soto}
  Electrical Engineering student at UC Santa Cruz. Power Supply and peripheral designer on the hardware team. 
\end{IEEEbiography}

\begin{IEEEbiography}
[{\includegraphics[width=1in,height=1.25in,clip,keepaspectratio]{profile/avalera.jpg}}]{August~Valera}
  Computer Engineering (Systems Programming) and Computer Science student at UC Santa Cruz. Lead UI/UX developer on the software team. Enjoys long walks on the beach.
\end{IEEEbiography}

\begin{IEEEbiography}{Jeffrey~Zheng}
  Electrical Engineering student at UC Santa Cruz.
\end{IEEEbiography}

% You can push biographies down or up by placing
% a \vfill before or after them. The appropriate
% use of \vfill depends on what kind of text is
% on the last page and whether or not the columns
% are being equalized.

%\vfill

% Can be used to pull up biographies so that the bottom of the last one
% is flush with the other column.
%\enlargethispage{-5in}


% that's all folks
\end{document}


